% !TEX root = ../../../under-spec-z.tex

\subsection{The Riemann hypothesis} % (fold)
\label{sub:the_riemann_hypothesis}

    One of the main motivations for trying to study $\ff_1$ is the hope that it will lead to a proof of the Riemann hypothesis.
    An excellent history of $\ff_1$, starting from Tit's original idea of `interpreting $S_n$ as the Chevalley group over the field of characteristic one' in \cite{Tits:1957tq} (aiming to explain an analogy from \cite{Steinberg:1951vd}), and ending with the relatively recent paper \cite{Connes:fkN5ADoK} (which deals with this zeta function approach) can be found in the introduction of \cite{Pena:2009vb}.

    In \cite{Kapranov:1996vk}, treating $\ff_1$ as any other finite field $\ff_p$ for $p$ prime, we think of its extension of degree $n$, denoted $\ff_{1^n}$, coming from adjoining the $n$th roots of unity.
    This idea is built upon in \cite[\S2.4]{Soule:2008wq}, where it is conjectured that
    \begin{equation*}
        \ff_{1^n}\otimes_{\ff_1}\zz = \zz[T]/(T^n-1).
    \end{equation*}
    Comparing this to the results in \cref{sub:diagonalisable_group_schemes}, we see that our definitions match, as we would hope.

    The Riemann zeta function, defined, for $s\in\cc$ with $\Re(s)>1$, as
    \begin{equation*}
        \zeta(s) = \sum_{n=1}^\infty\frac1{n^s}
    \end{equation*}
    is a particular example of an \emph{$L$-function}\footnote{
        Certain meromorphic functions on the complex plane (conjecturally) coming from the analytic continuation of an infinite series called an \emph{$L$-series}, which is associated to some mathematical object (for example, an \emph{Artin $L$-function} is associated to a linear representation of a Galois group).
        \emph{Dirichlet $L$-series} are of the form $\sum_{n=1}^\infty a_n/n^s$ where $(a_n)_{n\in\nn}$ is a complex sequence and $s\in\cc$.
    }.
    In fact, it is really the \emph{generalised Riemann hypothesis} -- which states that the non-trivial zeros of global\footnote{
        Defined as \emph{Euler products} of \emph{local zeta functions}.
    } $L$-functions lie on the line $\Re(z)=1/2$ in the complex plane -- that has profound implications across the whole of mathematics\footnote{
        Although nobody would likely turn their nose up at a proof of the Riemann hypothesis.
    }.
    The link between a conjecture by Emil Artin on Artin $L$-functions \cite{Artin:1923hi} and the Riemann hypothesis was pointed out by André Weil in a letter to Artin written on July 10th 1942\footnote{
        Listed as [1942] in the bibliography of \cite{Weil:2009vp}.
    }, and was later mentioned more concretely in \cite[4]{Weil:1947wd}\footnote{
        What follows is a translation by the author -- the original is in French.
    }:
    \begin{quotation}
        \emph{The Riemann hypothesis, after having lost hope of proving it by methods of the theory of functions, appears to us today in a new light, that shows it inseparable from the Artin conjecture on $L$-functions, these two problems being two sides of the same arithmetic-algebraic question, where the simultaneous study of all the cyclotomic extensions of a given number field will certainly play a vital role.}
    \end{quotation}
    Prior to this, in 1940\footnote{
        \cite{Weil:1940ub}, though he later showed that his result was independent of \emph{this ``transcendental'' theory} in \cite{Weil:1941tx}.
    }, Weil proved the \emph{Riemann hypothesis for curves over finite fields} by taking a smooth curve $C$ over a finite field $\ff_p$ and looking at the diagonal of $C\times_{\ff_p}C$.
    If we could think of $\zz$ as a smooth curve over some finite field (which seems natural since $\zz$ is of dimension one) then Weil's proof could hopefully be extended to a proof of the Riemann hypothesis.
    But $\zz$ is not an algebra over any field.
    However, one of the conjectured properties of $\ff_1$ is that $\zz$ is an $\ff_1$-algebra, and so we would be able to construct $\zz\times_{\ff_1}\zz$.

    Building on this idea, as well as previous conjectures by Artin \cite{Artin:1924ie}, led Weil in 1949 to the famous \emph{Weil conjectures} \cite{Weil:1949tx}, the proof of which provided the main impetus for Alexander Grothendieck's two decades of work building upon that of Jean-Pierre Serre.
    There were four conjectures: the \emph{rationality conjecture}, proved by Bernand Dwork in 1960 \cite{Dwork:1960ul}; the \emph{functional equation} and the \emph{Betti number} conjectures, proved by Grothendieck\footnote{
        Together with Michael Artin, Pierre Deligne, Michel Raynaud, Jean Girard, and many others.
    } in 1965 \cite{FormuledeLefschetz:1965vd,Cohomologieladique:1965uc}; and the \emph{Riemann hypothesis analogue}, proved by Deligne in 1974 \cite{Deligne:1974vf}.
    This last conjecture implies that, if $X$ is a smooth projective variety of dimension $n$ over $\ff_q$, then its \emph{local zeta function}
    \begin{equation*}
        \zeta_X(s)=\exp\left(\sum_{m=1}^\infty\frac{N_m q^{-sm}}{m}\right)
    \end{equation*}
    (where $N_m$ is the number of points of $X$ defined over the degree-$m$ extension $\ff_{q^m}$ of $\ff_q$) is such that its zeros and poles lie on the lines $\Re(s)=j/2$ for $j=1,2,\ldots,2n$.
    So, if we could realise $X=\spec\zz$ as a smooth projective variety of dimension $1$ over $\ff_1$, then $\zeta_X(s)=$ would be the Riemann zeta function, and a proof of the Riemann hypothesis would almost fall straight out, or so we might hope.

    \bigskip

    Obviously though, there are some obstructions, otherwise the Riemann hypothesis would have been solved by now.
    The main problem is that we have many definitions for $\ff_1$, and there have been many different ideas for what $\ff_1$-schemes should be (see \cite{Pena:2009vb}), but none of them have all of the properties that we need to prove the Riemann hypothesis.
    Even if we \emph{did} find some perfect definition, we would need to come up with new cohomology theorems, just as Grothendieck, Deligne, et al. did to solve the Weil conjectures.
    In fact, it seems as if solving the Riemann hypothesis is more a question of doing \emph{analytic} geometry over $\ff_1$ rather than \emph{algebraic} geometry over $\ff_1$.
    So where should we go from here?


% subsection the_riemann_hypothesis (end)
