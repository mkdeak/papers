% !TEX root = ../../under-spec-z.tex

\section{Three examples of relative geometry} % (fold)
\label{sec:three_examples_of_relative_geometry}

    \emph{We now present our first three examples of categories of relative schemes.} (\S3~\P1)

    \begin{note}
        As stated in \cref{ssub:conventions}, we adopt the convention that $0\in\nn$.
    \end{note}

    \subsection{Under $\spec\zz$} % (fold)
    \label{sub:under_}

        Instead of just restating the results of \cite[\S3.1--3.3,~4]{Toen:2005wxa} we try here to provide some motivation for these choices of examples.

        \bigskip

        We know that $\zz$ is the initial object in the category $\CRing$ of commutative rings; we said in \cref{sub:overview} that we would have to leave $\CRing$ in order to find schemes over bases that lie \emph{under} $\spec\zz$.
        If we remove all algebraic structure from $\CRing$ we end up with the category $\Set$ of sets, and we might worry that we have discarded too much structure to be able to define schemes any more.
        But $(\Set,\times,\singleton)$, where $\times$ is the cartesian product and $\singleton$ is a singleton, \emph{is} a cosmos\footnote{
            It is arguably the most tautological example of a cosmos.
        }, and so we can define schemes over $\Set$.
        Since $\Set$ is the prototype of a concrete category we can't really go down any further without becoming too far removed from our usual concept of algebraic structures.
        So we look at what (commutative associative and unital) structures we can find in between $\Set$ and $\CRing$.

        % We can endow a set with some commutative associative binary operation, which we might as well call addition, and pick an object to be the additive identity.
        % This results in a \emph{commutative monoid}.
        % All of the language that we have built so far heavily depends on the objects we study having \emph{commutative} and \emph{associative} operations, as well as an \emph{identity}.
        % So we have actually passed over a few less-strict structures -- \emph{semigroups} (where we don't require an identity) and \emph{magmas} (where we don't require associativity) -- but our tools are not general enough to deal with them.

        % Next we could go one of two ways: introducing additive inverses to our commutative monoid, thus making it an \emph{abelian group}; introducing some commutative multiplication with identity, thus making it a \emph{commutative semi\-ring}.
        % Either way, the only thing (of interest to us now) sitting above these two structure comes from combining the two, resulting in a \emph{commutative ring} -- this is where we stop, since classical algebraic geometry can deal with things from here on in.

        We can endow a set with some commutative associative binary operation, which we might as well call addition, and pick an object to be the additive identity -- this results in a \emph{commutative monoid}.
        Then we could go one of two ways: introducing additive inverses to our commutative monoid, making it an \emph{abelian group}; or introducing some commutative multiplication with identity, making it a \emph{commutative semi\-ring}.
        Either way, the only thing (of interest to us now) sitting above these two structure comes from combining the two, resulting in a \emph{commutative ring} -- this is where we stop, since classical algebraic geometry can deal with things from here.

        \begin{note}
            A very important thing to be aware of (that notation doesn't make entirely clear) is that when we speak about $\modc{A}$ for $A\in\comm{\ccat}$, \emph{this category depends on $\ccat$}.
            For example, if we take $\zz\in\comm{\Ab}=\CRing$ then $\modc{\zz}$ is the category of abelian groups, but if we consider \mbox{$\zz\in\comm{\Set}=\CMon$} then $\modc{\zz}$ is the category of \emph{monoid actions}, or \emph{$\zz$-actions} (sometimes confusingly written $\zz\hbox{-}\mathsf{Set}$).
            \emph{To avoid confusion in this section, we adopt the following (non-standard) notation:} we write $\modc{A}_\ccat$ to emphasise that \mbox{$A\in\comm{\ccat}$}.
        \end{note}

        Recalling \cref{df:comm-c,df:module-over-monoid}, the following pattern emerges:
        % \footnote{
        %     In fact, there is a much more general statement: when $\smc$ is a cosmos then $1$ is initial in $\comm{\ccat}$, \emph{and} $\modc{1}\equiv\ccat$.
        %     This follows straight from the definitions.
        %     A different way to approach this might be the following: there is a monoidal forgetful functor $\Ab\to\Set$ which lifts to a functor $\modc{A}_\Ab\to\modc{A}_\Set$, and by some abstract nonsense this has an adjoint, so we could hope to lift $X\in\modc{A}_\Set$ to some $X'\in\modc{A}_\Ab$.
        %     But for us, once we know what we are aiming for, it is much easier to start with some algebraic object $A$, and then pick some $\ccat$ with $A\in\comm{\ccat}$ such that $\modc{A}_\ccat$ is exactly what we want it to be.
        % }:
        \begin{equation*}
            \begin{tikzcd}
                & \modc{1}_\ccat & (1,\ccat) & \comm{\modc{1}_\ccat} &\\[-1.8em]
                \,\arrow[rrrr, dash] & & & & \,\\[-1.5em]
                & \Ab & (\zz,\Ab) & \CRing &\\
                & \CMon \arrow[u] & (\nn,\CMon) \arrow[u, "\text{introduce additive inverses}" description] & \CSemiring \arrow[u] &\\
                & \Set \arrow[u] & (\singleton,\Set) \arrow[u, "\text{introduce addition}" description] & \CMon \arrow[u] &
            \end{tikzcd}
        \end{equation*}

        It is true that $\modc{\singleton}_\Set=\Set$, but here lies the subtle issue: when we introduce addition we need our singleton to contain an additive identity \emph{and} an additive generator for $\nn$, but by definition these things must be distinct.
        That is, we want $0,1\in\singleton$ with $0\neq1$.
        Clearly, such a singleton doesn't exist, but in \cite{Tits:1957tq} Jacques Tits introduced the idea of the \emph{field with one element}: $\ff_1$.
        Although it is not a well-defined mathematical object\footnote{
            At least, not within our current definitions of algebraic objects.
        }, it is interesting to put aside such concerns and try to glean as much information from it as possible, especially when it arises in such a natural way as it does here\footnote{
            In fact, studying this $\ff_1$ is one of the main motivations for developing all of this abstract theory -- see \cref{sub:the_riemann_hypothesis}.
        }.

        \bigskip

        By definition, $1$ is initial in $\comm{\ccat}$ and so $\spec1$ is terminal in $\aff{\ccat}$.
        Thus we can think of schemes relative to $\modc{1}_\ccat$ as being schemes \emph{over} $\spec1$.
        It can be shown\footnote{
                These are relatively standard facts -- they can be found, for example, on the $n$Lab.
        } that $(\Ab,\otimes,\zz)$, $(\CMon,\otimes,\nn)$, and $(\Set,\times,\ff_1)$ are all cosmoses.
        In \cite[\S3.1--3.3]{Toen:2005wxa}, the authors work up to defining $\ff_1$-schemes, as well as a base change to $\zz$-schemes:
        \begin{equation}\label{eq:change-of-base-ff1-to-zz}
            (\blank\otimes_{\ff_1}\zz)\colon\schc{\ff_1}\to\schc{\zz}
        \end{equation}
        which acts on affine schemes by
        \begin{equation}\label{eq:ff1-to-zz-affine}
            \begin{alignedat}{3}
                (\blank\otimes_{\ff_1}\zz)\colon\aff{\modc{\ff_1}}&\to\aff{\modc{\nn}}&&\to\aff{\modc{\zz}}\\
                \op{\CMon}&\to\op{\CSemiring}&&\to\op{\CRing}\\
                M&\mapsto\nn[M]&&\mapsto\zz[M].
            \end{alignedat}
        \end{equation}
    
    % subsection under_ (end)







    \subsection{Diagonalisable group schemes} % (fold)
    \label{sub:diagonalisable_group_schemes}

        With \cref{eq:change-of-base-ff1-to-zz} we can try to generalise some constructions of classical schemes to $\schc{\nn}$ and $\schc{\ff_1}$.
        We now look at one such example, straight from \cite[\S4]{Toen:2005wxa}.

        \bigskip

        \emph{\Cite[\S XIV.3,~p.217]{Milne:2012wc} provides a general introduction to diagonalisable group schemes.}
        \emph{The basic idea is to define a functor}
        \begin{align*}
            \dd\colon\Ab&\to\Fun(\CRing,\Grp)\\
            M&\mapsto\Hom_\Grp(M,\blank^\times)
        \end{align*}
        \emph{Then $\dd(M)\colon R\mapsto\Hom_\Grp(M,R^\times)$ can be viewed as an `affine group'.}

        \bigskip

        An abelian group is also a commutative monoid, and $\aff{\modc{\ff_1}}=\op{\CMon}$.
        So for $M\in\Ab$ we can define
        \begin{equation*}
            \dd_{\ff_1}(M) = \spec M\in\schc{\ff_1}.
        \end{equation*}
        Now \cite[Proposition~3.3,~\S XIV.3,~p.217]{Milne:2012wc} tells us two things:
        \begin{enumerate}[(i)]
            \item $\dd(M)\cong\spec\zz[M]$;
            \item if $M$ is finitely generated then $\dd(M)$ is isomorphic to a finite product of copies of $\GG_m$ and $\mu_n$ (for various $n\in\nn$).
        \end{enumerate}
        If we define the affine $\ff_1$-scheme
        \begin{equation*}
            \spec(\ff_{1^n}) = \dd_{\ff_1}(\zz/n\zz)
        \end{equation*}
        then, using the above results and \cref{eq:ff1-to-zz-affine}, we see that
        \begin{enumerate}[(i)]
            \item $\dd_{\ff_1}(M)\otimes_{\ff_1}\zz \cong \dd(M)$;
            \item $\spec(\ff_{1^n})\otimes_{\ff_1}\zz \cong \mu_n \cong \dd(\zz/n\zz)$.
        \end{enumerate}
        So we can generalise $\dd(\blank)$ to $\dd_{\ff_1}(\blank)$ to obtain \emph{diagonalisable group schemes over $\ff_1$} that, after a change of base $\ff_1\to\zz$, agree with our existing notions of diagonalisable group schemes.
        By using a change of base $\ff_1\to\nn$ we can also recover a definition for $\nn$-schemes.

        This idea of $\spec\ff_{1^n}$ plays a major role in \cref{sub:the_riemann_hypothesis}.
    
    % subsection diagonalisable_group_schemes (end)

% section three_examples_of_relative_geometry (end)
