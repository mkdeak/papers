
% !TEX root = ../../../under-spec-z.tex

\subsection{Changes of base} % (fold)
\label{sub:changes_of_bases}

    % \emph{Throughout this section $(\ccat,\otimes,1_\ccat)$ and $(\dcat,\odot,1_\dcat)$ are both cosmoses.}

    The aim of this section is to provide some technical machinery that we will use in \cref{sec:three_examples_of_relative_geometry}, so we only state the main result of \cite[\S2.5]{Toen:2005wxa} and refer the reader back to there for more details.

    % \bigskip

    % Suppose we have some strong\footnote{
    %     \cite{Toen:2005wxa} uses the name \emph{monoidal functor} (as do many other sources) to refer to what is sometimes called a \emph{strong monoidal functor}.
    %     The choice of nomenclature is only important insofar as consistency; we decide to use `strong'.
    % } symmetric monoidal functor\footnote{
    %     That is, a functor that respects the symmetric monoidal structure of both $\ccat$ and $\dcat$.
    %     To be slightly more precise, the functor comes with an isomorphism $f(X)\odot f(Y)\congto f(X\otimes Y)$, natural in $X$ and $Y$, which respects the symmetry of $\otimes$ and $\odot$, and an isomorphism $1_\dcat\congto f(1_\ccat)$ satisfying certain coherence conditions.
    % } \mbox{$f\colon\ccat\to\dcat$} that has a right adjoint $g\colon\dcat\to\ccat$.
    % This right adjoint is not necessarily a strong monoidal functor, but it is a \emph{lax monoidal functor}\footnote{
    %     That is, we do have morphisms $g(X)\otimes g(Y)\to g(X\odot Y)$ and $1_\ccat\to g(1_\dcat)$ respecting coherence conditions and the symmetry of $\otimes$ and $\odot$, but they are not necessarily isomorphisms.
    % }, which turns out to be enough for $g$ to induce a functor on the categories of commutative monoids and on the category of modules.

    % By its symmetric monoidality, the functor $f\colon\ccat\to\dcat$ induces a functor $f\colon\aff{\ccat}\to\aff{\dcat}$.
    % This in turn induces a functor $f_!^\sim\colon\Shv(\aff{\ccat}\to\Shv(\aff{\dcat})$, given by precomposition by $g$ (and the use of \emph{sheafification}\footnote{
    %     Similar to classical algebraic geometry, the inclusion functor $\iota\colon\Shv(\aff{\ccat})\to\PShv(\aff{\ccat})$ admits a left adjoint $a\colon\PShv(\aff{\ccat})\to\Shv(\aff{\ccat})$ which we call the \emph{sheafification functor}.
    %     For (vastly many) more details, see \cite[\S III.5,~Theorem~1]{MacLane:1992uz}.
    % }).

    \begin{theorem}[\mbox{}{(Corollaire~2.22,~\S2.5,~p.27)}]\label{co:that-base-change-lemma-that-is-important}
        Let $(\ccat,\otimes,1_\ccat)$ and $(\dcat,\odot,1_\dcat)$ be cosmoses, and $f\colon\ccat\to\dcat$ be a strong symmetric monoidal functor\footnote{
            That is, a functor that respects the symmetric monoidal structure of both $\ccat$ and $\dcat$.
            To be slightly more precise, the functor comes with an isomorphism $f(X)\odot f(Y)\congto f(X\otimes Y)$, natural in $X$ and $Y$, which respects the symmetry of $\otimes$ and $\odot$, and an isomorphism \mbox{$1_\dcat\congto f(1_\ccat)$} satisfying certain coherence conditions.
            \cite{Toen:2005wxa} and many others use the name \emph{monoidal functor} to refer to what we (and some others) call a \emph{strong monoidal functor}.
            The choice of nomenclature is only important insofar as consistency; we agree to use `strong'.
        } with left adjoint $g\colon\dcat\to\ccat$.
        Define
        % \begin{align*}
        %     f_!\colon\PShv(\aff{\ccat})&\to\PShv(\aff{\dcat})\\
        %     F&\mapsto(F\circ g)
        % \end{align*}
        % and
        \begin{align*}
            f_!^\sim\colon\Shv(\aff{\ccat})&\to\Shv(\aff{\dcat})\\
            F&\mapsto(a\circ F\circ g)
        \end{align*}
        where $a$ is the \emph{sheafification}\footnote{
            Similar to classical algebraic geometry: the inclusion functor $\iota\colon\Shv(\aff{\ccat})\to\PShv(\aff{\ccat})$ admits a left adjoint $a\colon\PShv(\aff{\ccat})\to\Shv(\aff{\ccat})$ which we call the \emph{sheafification functor}.
            For (vastly many) more details, see \cite[\S III.5,~Theorem~1]{MacLane:1992uz}.
        } functor.
        Suppose that $g\colon\dcat\to\ccat$ is conservative and commutes with filtered colimits, and that, for every flat morphism $A\to B$ in $\comm{\ccat}$ and every $N\in \modc{f(A)}$, the natural morphism
        \begin{equation*}
            g(N)\otimes_A B\to g(N\odot_{f(A)}f(B))
        \end{equation*}
        is an isomorphism in $\modc{B}$.
        Then
        \begin{enumerate}[(i)]
            \item $f\colon\aff{\ccat}\to\aff{\dcat}$ is continuous in the Zariski topology;
            \item the functor $f_!^\sim\colon\Shv(\aff{\ccat})\to\Shv(\aff{\dcat})$ preserves the subcategories of schemes, and induces a functor (called the \emph{change of base functor})
                \begin{align*}
                    \Sch(\ccat)&\to\Sch(\dcat)\\
                    X&\mapsto X\otimes\dcat := f_!^\sim(X);
                \end{align*}
            \item we have an isomorphism
                \begin{equation*}
                    f_!^\sim(X)\cong f(X)
                \end{equation*}
                for every $X\in\ASch(\ccat)$.\qedhere
        \end{enumerate}
    \end{theorem}

    % \begin{proof}
    %     See \cite{Toen:2005wxa}.
    % \end{proof}

    \begin{note}[Notation]
        Say $(\mathcal{T},\otimes,1)$ is a cosmos and $A,B\in\comm{\mathcal{T}}$.
        Then, with $\ccat=\modc{A}$ and $\dcat=\modc{B}$, we write
        \begin{equation*}
            F_!^\sim(\blank)=(\blank\otimes_A B)\colon\Sch(\modc{A})\to\Sch(\modc{B}),
        \end{equation*}
        extending the definition of $(\blank\otimes_A B)$ from \cref{ssub:commutative_algebras}.
    \end{note}

    % The functor $F$ induces a functor $\comm{\ccat}\to\comm{\ccat}$, and thus $\aff{\ccat}\to\aff{\dcat}$, and this latter functor has a \emph{left}\footnote{
    %     Since $\aff{\ccat}$ is defined as the opposite category $\comm{\ccat}$, the handedness of the adjunction swaps when we descend to the induced functors.
    % } adjoint, induced by $G$.
    % That is, keeping the same notation for the induced functors as the original functors,
    % \begin{equation*}
    %     (G\colon\aff{\dcat}\to\aff{\ccat}) \dashv (F\colon\aff{\ccat}\to\aff{\dcat}).
    % \end{equation*}

    % Now, the functor $G\colon\aff{\dcat}\to\aff{\ccat}$ induces a functor on presheaf categories
    % \begin{equation*}
    %     G^*\colon\PShv(\aff{\ccat})\to\PShv(\aff{\dcat})
    % \end{equation*}
    % given, on objects, by precomposition:
    % \begin{equation*}
    %     G^*\colon X\mapsto (X\circ G),
    % \end{equation*}
    % and similarly for $F^*$.
    % We claim that there exists a left adjoint $G_!\dashv G^*$, and that it is given by $G_!=F^*$.
    % We do not prove this fact here, but it can be done in various ways\footnote{
    %     In our particularly nice situation (assuming smallness of categories, and using the fact that $\Set$ is, in particular, cocomplete) we can construct a left-adjoint $G_!$ for $G^*$ using the \emph{left Kan extension functor}.
    %     In fact, since $\Set$ is also complete, we also have a right adjoint $G^*\dashv G_*$, which turns out to be given by $G_*=F^*$.
    %     Full details can be found in \cite[\S X.3]{Lane:1998fe}.
    % }.

    % In fact, the adjuction $(F_!\dashv F^*)$ induces an adjuction $(F_!^\sim \dashv F^*)$ on the category of sheaves, where we use the \emph{sheafification\footnote{
    %     Similar to classical algebraic geometry, the inclusion functor $\iota\colon\Shv(\aff{\ccat})\to\PShv(\aff{\ccat})$ admits a left adjoint $a\colon\PShv(\aff{\ccat})\to\Shv(\aff{\ccat})$ which we call the \emph{sheafification functor}.
    %     For (vastly many) more details, see \cite[\S III.5,~Theorem~1]{MacLane:1992uz}.
    % } functor $a$} to define
    % \begin{equation*}
    %     F_!^\sim=(a\circ F_!)\colon\Shv(\aff{\ccat})\to\Shv(\aff{\dcat}).
    % \end{equation*}
    % Since $F$ has a right adjoint, and thus commutes with colimits, $F_!^\sim$ commutes in particular with finite limits.
    % An adjuction $(f\dashv g)$ between sites\footnote{
    %     It should really be an adjuction between \emph{toposes}, though we don't have time to discuss this notion properly.
    % } is said to be a \emph{geometric morphism} if $f$ commutes with finite limits, and we have just shown that $(F_!^\sim \dashv F^*)$ is such an adjuction.
    
    % \begin{definition}[Continuous in a topology]\label{df:continuous-in-topology}
    %     A functor $F\colon\aff{\ccat}\to\aff{\dcat}$ is said to be \emph{continuous in the Zariski} (resp. \emph{fpqc}) \emph{topology} if the induced functor
    %     \begin{equation*}
    %         F^*\colon\PShv(\aff{\dcat})\to\PShv(\aff{\ccat})
    %     \end{equation*}
    %     (as defined above) preserves the subcategory of Zariski (resp. fpqc) sheaves.
    % \end{definition}

    % Note that this definition is different from the definition of a continuous functor in the sense of preserving small limits.
    % Because of this we will always ensure to talk of a functor being continuous \emph{in a topology} if we mean continuous in the sense of \cref{df:continuous-in-topology}.

    % \begin{lemma}[\mbox{}{Proposition~2.21,~\S2.5,~p.26}]
    %     Using the above notation,
    %     \begin{enumerate}[(i)]
    %         \item $F\colon\aff{\ccat}\to\aff{\dcat}$ preserves monomorphisms and finite limits;
    %         \item if $F\colon\aff{\ccat}\to\aff{\dcat}$ is continuous in the fpqc topology and $G\colon\dcat\to\ccat$ commutes with filtered colimits, then $F\colon\aff{\ccat}\to\aff{\dcat}$ is also continuous in the Zariski topology.\qedhere
    %     \end{enumerate}
    % \end{lemma}

    % \begin{proof}
    %     This is largely just unpacking definitions.
    %     See \cite{Toen:2005wxa}.
    % \end{proof}

    % \begin{corollary}[\mbox{}{(Corollaire~2.22,~\S2.5,~p.27)}]\label{co:that-base-change-lemma-that-is-important}
    %     Using the above notation, suppose that the functor $G\colon\dcat\to\ccat$ is conservative and commutes with filtered colimits.
    %     Suppose further that, for every flat morphism $A\to B$ in $\comm{\ccat}$ and every $N\in \modc{F(A)}$, the natural morphism
    %     \begin{equation*}
    %         G(N)\otimes_A B\to G(N\odot_{F(A)}F(B))
    %     \end{equation*}
    %     is an isomorphism in $\modc{B}$.
    %     Then
    %     \begin{enumerate}[(i)]
    %         \item $F\colon\aff{\ccat}\to\aff{\dcat}$ is continuous in the Zariski topology;
    %         \item the functor $F_!^\sim\colon\Shv(\aff{\ccat})\to\Shv(\aff{\dcat})$ preserves the subcategories of schemes, and induces a functor
    %             \begin{align*}
    %                 \Sch(\ccat)&\to\Sch(\dcat)\\
    %                 X&\mapsto X\otimes\dcat := F_!^\sim(X);
    %             \end{align*}
    %         \item we have an isomorphism
    %             \begin{equation*}
    %                 F_!^\sim(X)\cong F(X)
    %             \end{equation*}
    %             for every $X\in\ASch(\ccat)$.\qedhere
    %     \end{enumerate}
    % \end{corollary}

    % \begin{proof}
    %     See \cite{Toen:2005wxa}.
    % \end{proof}

    % \begin{note}[Notation]
    %     Say $(\mathcal{T},\otimes,1)$ is a cosmos and $A,B\in\comm{\mathcal{T}}$.
    %     Then, with $\ccat=\modc{A}$ and $\dcat=\modc{B}$, we write
    %     \begin{equation*}
    %         F_!^\sim(\blank)=(\blank\otimes_A B)\colon\Sch(\modc{A})\to\Sch(\modc{B}),
    %     \end{equation*}
    %     extending the definition of $(\blank\otimes_A B)$ from \cref{ssub:commutative_algebras}.
    % \end{note}

    
    As we would (very much) hope, the change of base functor is functorial: it doesn't matter in which order we change base and sheafify.
    In particular, if $X=\spec A\in\aff{\ccat}$ for some $A\in\comm{\ccat}$ then \cref{co:that-base-change-lemma-that-is-important}~(iii) tells us\footnote{
        \cite[\S2.5,~\P-1]{Toen:2005wxa}
    } that
    \begin{equation*}
        f_!^\sim(\spec A)\cong\spec f(A).
    \end{equation*}
    Thus, using the Hom-set isomorphism from the adjunction $(f\dashv g)$,
    \begin{align*}
        f_!^\sim(\spec A)\colon\comm{\dcat}&\to\Set\\
        B&\mapsto\Hom(f(A),B)\cong\Hom(A,g(B)).
    \end{align*}
    We think of $\spec f(A)$ as\footnote{
        Again, recall \cref{df:affine-schemes-general}.
    }
    \begin{align*}
        \spec f(A) \quad\mlq=\mrq\quad &\Hom_{\aff{\dcat}}(\blank,\spec f(A))\\
        =&\Hom_{\comm{\dcat}}(f(A),\blank).
    \end{align*}
    So, remembering \cref{ssub:motivating_example}, $f_!^\sim(\spec A)$ gives the \emph{functor of points of $A$}.

    % As we would (very much) hope, the change of base functor is functorial: spelling out \cref{co:that-base-change-lemma-that-is-important}~(iii) below tells us that it doesn't matter in which order we change base and sheafify.

    % \begin{translation}{2.5}{-1}
    %     For a commutative monoid $A\in\comm{\ccat}$ we have a natural isomorphism
    %     \begin{equation*}
    %         F_!^\sim(\spec A)\cong\spec F(A).
    %     \end{equation*}
    %     Indeed, by definition the functor $F_!^\sim$ is the composition of $F_!$ (defined on presheaves) and $a$ (the sheafification functor).
    %     It is easy to see that $F_!(\spec A)\cong\spec F(A)$ as \emph{presheaves}.
    %     Since presheaves that are representable by affine schemes are sheaves, we see that $F_!^\sim(\spec A)\cong\spec F(A)$.
    %     This tells us that
    %     \begin{align*}
    %         F_!^\sim(\spec A)\colon\comm{\dcat}&\to\Set\\
    %         B&\mapsto\Hom_{\comm{\ccat}}(A,G(B)).
    %     \end{align*}
    %     for all $\spec A\in\aff{\ccat}$.
    % \end{translation}

    % TJH what does this mean?
    % TJH recall functor of points ($\Hom(A,\blank)$) and use Yoneda \emph{lemma}
    % ...
    % \begin{align*}
    %     &F_!^\sim(\spec A)(\blank)\\
    %     \cong &\Hom_{\comm{\dcat}}(F(A),\blank)\\
    %     \cong &\Hom_{\PShv(\aff{\dcat})}(\spec \blank,\spec F(A))\in\PShv(\aff{\dcat})
    % \end{align*}
    % \textbf{\emph{...so what?}}
    % the second one doesn't need Yoneda, but looks like the functor of points approach for $F(A)$?

% subsection changes_of_bases (end)
