% PREAMBLE
\documentclass[10pt]{article}


%% Packages

\usepackage{amsmath,amssymb,amsthm}
\usepackage[capitalize]{cleveref}
\usepackage{palatino}
\usepackage[margin=1.2in]{geometry}
\usepackage[utf8]{inputenc}
\usepackage[T1]{fontenc}
\usepackage{tikz-cd}
\usepackage{wrapfig}

\usepackage{chngcntr}
\counterwithin{figure}{subsubsection}

%% Title

\title{Schemes: a cheat sheet}
\author{Timothy Hosgood}

%% Operators

\DeclareMathOperator{\Op}{Op}
\DeclareMathOperator{\colim}{colim}
\DeclareMathOperator{\eq}{eq}
\DeclareMathOperator{\Ker}{Ker}
\DeclareMathOperator{\Coker}{Coker}
\DeclareMathOperator{\coker}{coker}
\DeclareMathOperator{\Supp}{Supp}
\DeclareMathOperator{\Spec}{Spec}
\DeclareMathOperator{\Proj}{Proj}
\DeclareMathOperator{\V}{\mathbb{V}}
\DeclareMathOperator{\I}{\mathbb{I}}
\DeclareMathOperator{\D}{D}

\renewcommand{\Im}{\mathrm{Im}\,}

\newcommand{\dirlim}[1]{\underset{#1}{\lim\limits_{\longrightarrow}}\,}
\newcommand{\invlim}[1]{\underset{#1}{\lim\limits_{\longleftarrow}}\,}

%% Shortcuts

\newcommand{\Set}{\mathsf{Set}}
\newcommand{\Ab}{\mathsf{Ab}}
\newcommand{\CRing}{\mathsf{CRing}}
\newcommand{\Top}{\mathsf{Top}}
\newcommand{\Etale}{\mathsf{\acute{E}tal\acute{e}}\,}
\newcommand{\Cover}{\mathsf{Cover}}
\newcommand{\lc}{\mathrm{l.c.}}
\newcommand{\Mod}[1]{{#1}\hbox{-}\mathsf{Mod}}
\newcommand{\ccat}{\mathcal{C}}
\newcommand{\fsheaf}{\mathcal{F}}
\newcommand{\gsheaf}{\mathcal{G}}
\newcommand{\pre}{\text{pre}}
\newcommand{\id}{\mathrm{id}}
\renewcommand{\O}{\mathcal{O}}
\newcommand{\dual}{\vee}
\newcommand{\Z}{\mathbb{Z}}
\newcommand{\Hom}{\mathrm{Hom}}
\newcommand{\sHom}{\underline{\mathrm{Hom}}\,}
\newcommand{\congto}{\xrightarrow{\sim}}
\newcommand{\prid}{\mathfrak{p}}
\newcommand{\maid}{\mathfrak{m}}
\newcommand{\nilr}{\mathfrak{N}}
\newcommand{\A}{\mathbb{A}}
\renewcommand{\P}{\mathbb{P}}
\newcommand{\ideal}{\triangleleft}

% DOCUMENT

\begin{document}

    \maketitle
    \begin{abstract}
        This is a cheat sheet to elementary scheme theory, following Ravi Vakil's wonderful notes/book \emph{The Rising Sea}.
        It it a constant work-in-progress, but unfortunately not at the top of my to-do list at the moment.
        Hopefully one day I'll be able to come back and finish this off.
    \end{abstract}
    \tableofcontents
    
    \section{Conventions}
    
        \begin{itemize}
            \item Any references of the form (1.1.A) are to Vakil's \emph{The Rising Sea}
            \item All rings have an identity with $1\neq0$ and are commutative unless otherwise specified.
            \item Inclusions of sets (groups, rings, etc.) are \textit{not} assumed to be strict: $U\subset V$ means $U\subseteq V$; if $U\neq V$ then we write $U\subsetneq V$.
            \item We write $\{*\}$ to mean a set with one element.
            \item Maps between topological spaces are assumed to be continuous unless otherwise specified.
            \item In an abelian category, given some morphism $\varphi\colon x\to y$, we write $\Ker\varphi$ to mean the object in the kernel pair, and $\ker\varphi$ to mean the morphism, i.e. $\ker\varphi\colon\Ker\varphi\to x$; we do similarly for the cokernel and image.
            \item We write $\ccat/X$ to be the slice category, i.e. its objects are pairs $(Y,f)$ where $Y\in\ccat$ and $f\colon Y\to X$.
            \item Given some topological space $X$ we write $\Op(X)$ to be the category of open sets of $X$, whose morphisms are given by inclusions.
            \item We sometimes write `iff' to mean `if and only if'.
        \end{itemize}
    
    \section{Sheaves}
    
        \subsection{Definitions}
    
            Here we write $X$ to mean an arbitrary topological space, $\fsheaf$ an arbitrary sheaf on $X$, $p$ an arbitrary point in $X$, $U$ an arbitrary open set in $X$, etc.
            
            \subsubsection{Preliminary definitions}
    
                \begin{itemize}
                    \item \textit{Presheaf}: a contravariant functor $\fsheaf\colon\Op(X)\to\ccat$ (where $\ccat$ is $\Set$, $\Ab$, $\CRing$, or some similar category).
                    \item \textit{Section}: an element of $\fsheaf(U)$ for some $U\in\Op(X)$ (and if $U=X$ then the element is called a \textit{global section}); \textbf{n.b.} depending on the context a section can also mean the right inverse of a continuous map, i.e. a section of $f\colon X\to Y$ is $\mu\colon Y\to X$ such that $f\circ\mu = \id_Y$ (and a left inverse is called a \textit{retraction}).
                    \item \textit{Sheaf}: a presheaf $\fsheaf$ satisfying the \textit{identity} and \textit{gluability} axioms for all open sets $U\in\Op(X)$ and covers $\{U_i\}_{i\in I}$ of $U$:
                        \begin{itemize}
                            \item \textit{Identity/locality}: if $f_1,f_2\in\fsheaf(U)$ agree in all $\fsheaf(U_i)$ then $f_1=f_2$;
                            \item \textit{Gluability}: if $f_i\in\fsheaf(U_i)$ agree in all $\fsheaf(U_i\cap U_j)$ then there exists $f\in\fsheaf(U)$ such that $f$ restricts to $f_i$ in $\fsheaf(U_i)$.
                        \end{itemize}
                        Equivalently, we can ask that $\fsheaf$ satisfies the following exactness requirement:
                        \[\fsheaf(U) = \eq\left(\prod\fsheaf(U_i)\rightrightarrows\prod\fsheaf(U_i\cap U_j)\right).\]
                        We write $\ccat_X$ to mean the \textit{category of sheaves in $\ccat$ on $X$}, where $\ccat$ is one of $\Set$, $\Ab$, $\CRing$, etc.
                    \item \textit{Separated presheaf}: a presheaf satisfying only the identity axiom.
                    \item \textit{Stalk}: given a point $p\in X$ we define the stalk of $\fsheaf$ at $p$ by
                        \begin{align*}
                            \fsheaf_p &= \dirlim{U\ni p}\fsheaf(U)\\
                            &= \big\{(f,U) \mid p\in U, f\in\fsheaf(U)\big\}\big/\sim
                        \end{align*}
                        where $(f,U)\sim(g,V)$ if there exists $W\subset U\cap V$ such that the restriction of $f$ and $g$ to $W$ agree.
                    \item \textit{Germ}: an element of a stalk; given a particular $f\in\fsheaf(U)$ with $p\in U$ we say \textit{the germ of $f$ at $p$} to mean the image $f_p$ of $f$ in $\fsheaf_p$.
                    \item \textit{Compatible germs}: an element \[(s_p)_{p\in U}\in\prod_{p\in U}\fsheaf_p\] consists of \textit{compatible germs} if there exists an open cover $\{U_i\}$ of $U$ and sections $f_i\in\fsheaf(U_i)$ such that for all $p\in U_i$ the germ of $f_i$ at $p$ is $s_p$.
                     (\textbf{n.b.} the notation can be confusing: $(s_p)_{p\in U}$ \textit{doesn't} mean we take one section $s$ and consider its germ at all points (otherwise this definition would be redundant!) but instead the section \textit{varies} with the point at which we take the germ, so perhaps something more cumbersome like $(s^{(p)}_p)_{p\in U}$ would be a better notation); see (2.4.C) in \cref{subsection:sheaves-facts} and all of \cref{subsection:espace-etale} for more information).
                    \item \textit{Sheafification}: a morphism $a\colon\fsheaf\to\fsheaf^a$ of presheaves such that $\fsheaf^a$ is a sheaf and, for any sheaf $\gsheaf$ and presheaf morphism $g\colon\fsheaf\to\gsheaf$ there \textit{exists} a \textit{unique} sheaf morphism $f\colon\fsheaf^a\to\gsheaf$ such that the following diagram commutes:
                        \[\begin{tikzcd}
                            \fsheaf \ar{r}{a} \ar{dr}[swap]{g} & \fsheaf^a \ar{d}{f}\\
                            & \gsheaf
                        \end{tikzcd}\]
                        We can construct this explicitly by using compatible sections:
                        \[\fsheaf^a(U) = \left\{(f_p)_{p\in U}\in\prod\nolimits_{p\in U}\fsheaf_p \mid \forall p\in U\,\, \exists V_p\in\Op(U),\,s\in\fsheaf(V_p)\text{ s.t. }p\in V_p\text{ and }s_q=f_q\,\forall q\in V\right\}\]
                    \item \textit{Support of a sheaf}: when $\fsheaf\in\ccat_X$ we define \[\Supp\fsheaf = \{p\in X \,\big|\, \fsheaf_p\neq t_\ccat\}\]
                        where $t_\ccat$ is the terminal object in $\ccat$ (so $t_\Set=\{*\}$, $t_\Ab=t_\CRing=0$).
                    \item \textit{Support of a section}: when $\fsheaf\in\ccat_X$ and $s$ is a global section we define \[\Supp(s) = \{p\in X\mid s_p\neq t_\ccat\in\fsheaf_p\}\]
                        where $t_\ccat$ is the terminal object in $\ccat$.
                \end{itemize}
                
            \subsubsection{Ringed spaces}
                
                \begin{itemize}
                    \item \textit{Ringed space}: a pair $(X,\O_X)$ where $\O_X$ is a sheaf of rings on $X$, called the \textit{structure sheaf}.
                    \item \textit{Functions on a ringed space}: the elements $f\in\O_X(U)$ for an open $U\subset X$ are called \textit{functions} on $U$.
                    \item \textit{Locally ringed space}: a ringed space where every stalk is a local ring (i.e. has a unique maximal ideal).
                    \item \textit{Sheaf of $\O_X$-modules}: a sheaf $\fsheaf$ of abelian groups on $X$ such that each $\fsheaf(U)$ is an $\O_X(U)$-module in a functorial way (\textbf{n.b.} this also makes each stalk $\fsheaf_p$ an $\O_{X,p}$-module).
                    \item \textit{Sheaf Hom}: the sheaf $\sHom_{\ccat_X}(\fsheaf,\gsheaf)$ defined by
                        \[\sHom_{\ccat_X}(\fsheaf,\gsheaf)(U)=\Hom_{\ccat_X}(\fsheaf|_U,\gsheaf|_U)\]
                        where $\ccat$ is one of $\Set$, $\Ab$, $\CRing$, etc. (\textbf{n.b.} this is \textbf{not} the same as $\Hom_\Set(\fsheaf(U),\gsheaf(U))$: the former is the collection of natural transformations, i.e. sheaf morphisms; the latter is the collection of set morphisms).
                    \item \textit{Dual of an $\O_X$-module}: $\fsheaf^\dual = \sHom_{\Mod{\O_X}}(\fsheaf,\O_X)$.
                \end{itemize}
                    
            \subsubsection{Abelian category structure}    
                
                \begin{itemize}
                    \item \textit{Subsheaves and quotient sheaves}: a sheaf $\fsheaf$ is a \textit{subsheaf} (resp. \textit{quotient sheaf}) of another sheaf $\gsheaf$ if there exists a monomorphism $\fsheaf\to\gsheaf$ (resp. epimorphism $\gsheaf\to\fsheaf$).
                    \item \textit{Presheaf (co)kernel}: given a morphism of \textit{presheaves} $\varphi\colon\fsheaf\to\gsheaf$ we define presheaves by 
                        \begin{align*}
                            (\Ker_\pre\varphi)(U) &= \Ker\varphi(U);\\
                            (\Coker_\pre\varphi)(U) &= \Coker\varphi(U).\\
                        \end{align*}
                        These satisfy the universal properties of (co)kernels in the category of presheaves on $X$.
                    \item \textit{Sheaf (co)kernel}: the presheaf kernel is already a sheaf: $\Ker\varphi=\Ker_\pre\varphi$; the presheaf cokernel needs to be sheafified: $\Coker\varphi=(\Coker_\pre\varphi)^a$.
                    \item \textit{Sheaf image}: $\Im\varphi=(\Im_\pre\varphi)^a$.
                \end{itemize}
                
            \subsubsection{Direct and inverse images}
            
                Let $\pi\colon X\to Y$ be a continuous map of topological spaces, $\fsheaf$ a sheaf of abelian groups on $X$, and $\gsheaf$ a sheaf of abelian groups on $Y$.
            
                \begin{itemize}
                    \item \textit{Direct image/pushforward}: define a sheaf on $Y$ by
                        \[\pi_*\fsheaf(V) = \fsheaf(\pi^{-1}(V)).\]
                        So $\pi_*\colon\Ab_X\to\Ab_Y$.
                    \item \textit{Inverse image:} define a \textbf{presheaf} on $X$ by \[\pi_\pre^{-1}\gsheaf(U) = \dirlim{V\supset\pi(U)}\gsheaf(V)\]
                        and then sheafify to get the inverse image sheaf: \[\pi^{-1}\gsheaf=(\pi_\pre^{-1}\gsheaf)^a.\]
                        So $\pi^{-1}\colon\Ab_Y\to\Ab_X$.
                    \item \textit{Extension by zero}: let $i\colon U\to Y$ be the inclusion of an \textbf{open} set and define the \textbf{presheaf}
                        \[(i_!^\pre\fsheaf)(W)=\begin{cases}
                            \fsheaf(W) &\text{if }W\subset U;\\
                            0 &\text{otherwise},
                        \end{cases}\]
                        and then sheafify to get the extension by zero:
                        \[i_!\fsheaf=(i_!^\pre\fsheaf)^a.\]
                        So $i_!\colon\Mod{\O_U}\to\Mod{\O_Y}$.
                \end{itemize}
            
            \subsubsection{Common examples}
            
                Let $S\in\Set$ be an arbitrary set.
            
                \begin{itemize}
                    \item \textit{Skyscraper sheaf}:
                        \[i_{p,*}S(U)=\begin{cases}
                            S &\text{if }p\in U;\\
                            \{*\} &\text{if }p\not\in U.
                        \end{cases}\]
                    \item \textit{Constant presheaf}: $\underline{S}_\pre(U)=S$.
                    \item \textit{Constant sheaf}: $\underline{S}(U)=\{\text{locally constant }f\colon U\to S\}$ (\textbf{n.b.} this is the sheafification of the constant presheaf)
                \end{itemize}

        \subsection{Facts}\label{subsection:sheaves-facts}
        
            \subsubsection{Presheaves}
            
                \textit{Motto: \textbf{presheaves} work well on the level of \textbf{sections}}
            
                \begin{itemize}
                    \item[(2.3.H)] A sequence of \textbf{presheaves} $0\to\fsheaf_1\to\ldots\to\fsheaf_n\to0$ is exact if and only if it is exact on \textit{all sections}, i.e. if and only if $0\to\fsheaf_1(U)\to\ldots\to\fsheaf_n(U)\to0$ is exact for all $U\in\Op(X)$.
                \end{itemize}
                
            \subsubsection{Sheafification}
            
                \begin{itemize}
                    \item[(2.4.L)] Sheafification is left-adjoint to the forgetful functor from sheaves to presheaves.
                    \item[(2.4.8)] Sheafification is equivalent to taking the sheaf of sections of the espace étalé of a presheaf (see \cref{subsection:espace-etale}).
                \end{itemize}
                
            \subsubsection{Direct and inverse images}\label{subsubsection:direct-inverse-images}
            
                \begin{itemize}
                    \item[(2.6.B)] $(\pi^{-1}\dashv\pi_*)$
                    \item[(2.5.G), (2.6.E)] Direct images are \textbf{left-exact} functors; inverse images and extensions by zero are \textbf{exact} functors.
                    \item[(2.6.3)] If $i\colon\{p\}\to Y$ is the inclusion of a point and $\gsheaf$ is a sheaf of abelian groups on $Y$ then $i^{-1}\gsheaf$ is the stalk $\gsheaf_p$.
                        Thus `stalks are left adjoint to skyscrapers'.
                    \item[(2.6.D)] If $i\colon U\to Y$ is the inclusion of an \textbf{open} set and $\gsheaf$ is a sheaf of abelian groups on $Y$ then $i^{-1}\gsheaf$ is the restriction $\gsheaf|_U$.
                    \item[(2.6.F)] If $j\colon Z\to Y$ is the inclusion of a \textbf{closed} set and $\fsheaf$ is a sheaf of sets on $Z$ then \[(j_*\fsheaf)_p=\begin{cases}
                            \fsheaf_p &\text{if }p\in Z;\\
                            \{*\} &\text{if }p\not\in Z.
                        \end{cases}\]
                        If $\gsheaf$ is a sheaf on $Y$ \textbf{such that} $\Supp\gsheaf\subset Z$ then $\gsheaf\congto i_*i^{-1}\gsheaf$.
                        Thus `a sheaf supported on a closed subset can be considered a sheaf on that subset.'
                    \item[(2.6.G)] If $i\colon U\to Y$ is the inclusion of an \textbf{open} set and $\fsheaf$ is an $\O_U$-module, then \[(i_!\fsheaf)_p=\begin{cases}
                            \fsheaf_p &\text{if }p\in U;\\
                            0 &\text{if }p\not\in U.
                        \end{cases}\]
                    \item[(2.6.G)] If $i\colon U\to Y$ is the inclusion of an \textbf{open} set, $j\colon Z\to Y$ is the inclusion of its \textbf{complement}, and $\gsheaf$ is an $\O_Y$-module, then the following sequence is short exact:
                        \[0 \to i_!i^{-1}\gsheaf \to \gsheaf \to j_*j^{-1}\gsheaf \to 0.\]
                        Further, $(i_!\dashv i^{-1})$.
                \end{itemize}
            
            \subsubsection{Sheaves}
            
                \textit{Motto: \textbf{sheaves} work well on the level of \textbf{stalks}}
        
                \begin{itemize}
                    \item[(2.2.C)] $\fsheaf(\cup_i U_i) = \invlim{}\fsheaf(U_i)$.
                    \vspace{-1em}
                    \item[(2.3.D)] $\sHom_{\Set_X}\big(\underline{\{p\}},\fsheaf\big)\cong\fsheaf$;\\
                        $\sHom_{\Ab_X}\big(\underline{\Z},\fsheaf\big)\cong\fsheaf$;\\
                        $\sHom_{\Mod{\O_X}}\big(\O_X,\fsheaf\big)\cong\fsheaf$.
                    \item[(2.4.B)] The support $\Supp(s)$ of a section is a closed subset of $X$.
                    \item[(2.4.A), (2.4.C)] Sections of a \textbf{sheaf} (or even a separated presheaf) are determined by germs, i.e. the natural map \[\fsheaf(U)\longrightarrow\prod_{p\in U}\fsheaf_p\] is injective.
                        Further, the image of this map is exactly the set of compatible sections.
                    \item[(2.4.D)] Morphisms of \textbf{sheaves} are determined by stalks, i.e. if $\varphi,\psi\colon\fsheaf\to\gsheaf$ and $\varphi_p=\psi_p$ for all $p\in X$ then $\varphi=\psi$.
                    \item[(2.4.E)] Isomorphisms of \textbf{sheaves} are determined by stalks, i.e. if $\varphi\colon\fsheaf\to\gsheaf$ induces isomorphisms $\varphi_p\colon\fsheaf_p\congto\gsheaf_p$ for all $p\in X$ then $\fsheaf\cong\gsheaf$ (\textbf{n.b.} this does \textbf{not} say that if $\fsheaf_p\cong\gsheaf_p$ for all $p$ then $\fsheaf\cong\gsheaf$).
                    \item[(2.4.N), (2.4.O)] For a morphism $\varphi\colon\fsheaf\to\gsheaf$ of \textbf{sheaves} the following are equivalent:
                        \begin{itemize}
                            \item $\varphi$ is a monomorphism in the category of sheaves;
                            \item $\varphi$ is injective on the level of stalks;
                            \item $\varphi$ is injective on the level of sections.
                        \end{itemize}
                        The corresponding statements are also equivalent
                        \begin{itemize}
                            \item $\varphi$ is an epimorphism in the category of sheaves;
                            \item $\varphi$ is surjective on the level of stalks.
                        \end{itemize}
                        (\textbf{n.b.} there is no third statement for epimorphisms; see (2.5.F) for more).
                    \item[(2.5.A)] Stalks and (co)kernels of \textbf{sheaves} commute, i.e. $\Ker(\fsheaf\to\gsheaf)_p\cong\Ker(\fsheaf_p\to\gsheaf_p)$.
                    \item[(2.5.D)] A sequence of \textbf{sheaves} $0\to\fsheaf_1\to\ldots\to\fsheaf_n\to0$ is exact if and only if it is exact on \textit{all stalks}, i.e. if and only if $0\to(\fsheaf_1)_p\to\ldots\to(\fsheaf_n)_p\to0$ is exact for all $p\in X$.
                    \item[(2.5.F), (2.5.G), (2.5.H)] The following functors are all left exact:
                        \begin{itemize}
                            \item[($\Ab_X\to\Ab$)] taking sections over $U$ for any $U\in\Op(X)$;
                            \item[($\Ab_X\to\Ab_Y$)] the direct image $\pi_*$ for any continuous $\pi\colon X\to Y$ (see \cref{subsubsection:direct-inverse-images});
                            \item[($\Ab_X\to\Ab_X$)] the covariant $\sHom(\fsheaf,-)$ and the contravariant $\sHom(-,\fsheaf)$.
                        \end{itemize}
                    \item[(2.5.2), (2.5.I)] $\Ab_X$ and $\Mod{\O_X}$ (where $(X,\O_X)$ is a ringed space) are abelian categories.
                \end{itemize}
            
        \subsection{Étalé spaces and sheaves}\label{subsection:espace-etale}
        
                \textbf{n.b.} there is a difference between \textit{étalé} (from the verb \textit{étaler}, roughly meaning `spread out') and \textit{étale} (a reasonably obscure nautical term: «l'étale» is the moment in between changing tides where the water is `slack').
                    There is often confusion between the two terms, and it is common for them to be mixed up.
                    The correct notions are of an « espace étalé » (i.e. an \textit{étalé space}) but a « morphisme étale » (i.e. an \textit{étale morphism}).
                    We refrain from using the phrase \textit{étalé morphism} to describe the map in an étalé space and instead just use the phrase `local homeomorphism' (even though an étale morphism between two topological spaces is often defined to be exactly a local homeomorphism).
        
            \subsubsection{Étalé spaces and sheaves}
                
                \begin{figure}[h!]
                    \centering
                    \begin{tikzcd}[column sep=huge, row sep=huge]
                        \Top/X
                            \ar[bend left=5,dashed]{dr}[very near start,description]{\Gamma}
                        &\Set_X^\pre
                            \ar[bend right=5,dashed]{dl}[very near start,description]{\Lambda}\\
                        %%
                        \Etale(X)
                            \ar[bend left=20]{r}[description]{\Gamma}
                            \ar[dashed,tail]{u}
                            \ar[phantom]{r}[description]{\perp}
                        &\Set_X
                            \ar[bend left=20]{l}[description]{\Lambda}
                            \ar[dashed,tail]{u}\\
                        %%
                        \Cover(X)
                            \ar[bend left=10]{r}[description]{\Gamma}
                            \ar[dashed,tail]{u}
                        &\Set_X^\lc
                            \ar[bend left=10]{l}[description]{\Lambda}
                            \ar[dashed,tail]{u}
                    \end{tikzcd}
                    \caption{The adjunctions given by $\Gamma$ and $\Lambda$}
                    \label{figure:adjunction-diagram-gamma-lambda}
                \end{figure}  
                                   
                \begin{itemize}
                    \item \textit{Local homeomorphism}: a (continuous) map $p\colon E\to X$ such that for all $e\in E$ there exists some open neighbourhood $U_e\subset E$ of $e$ such that $p|_{U_e}\colon U_e\to p(U_e)$ is a homeomorphism
                        (\textbf{n.b.}  this is weaker than the notion of a \textit{covering space} – see \cref{subsubsection:covering-spaces-locally-constant-sheaves}).
                    \item \textit{Étalé space over $X$}: a pair $(E,p)$ where $E$ is a topological space and $p\colon E\to X$ is a local homeomorphism.
                        We write the full subcategory of $\Top/X$ consisting of étalé spaces over $X$ as $\Etale(X)$.
                    \item \textit{Associated presheaf}: define the functor $\Gamma\colon\Top/X\to\Set_X^\pre$ by sending a slice to the sheaf of sections, i.e.
                        \begin{equation*}
                            \Gamma\left(E\xrightarrow{p}X\right)\big(U\big) = \left\{\sigma\colon U\to E \mid p\circ\sigma =
                            id_U\right\}.
                        \end{equation*}
                        It turns out that the image under $\Gamma$ of \textit{any} slice $(E\xrightarrow{p}X)\in\Top/X$ is in fact a sheaf.
                    \item \textit{Associated topological space}: define the functor $\Lambda\colon\Set_X^\pre\to\Top/X$ by sending a presheaf to the space consisting of the disjoint union (the coproduct in $\Set$) of all the stalks, i.e.
                        \begin{equation*}
                            \Lambda(\fsheaf) = \coprod_{x\in X}\fsheaf_x
                        \end{equation*}
                        with the topology \textit{generated} form the basis of the germs of sections on open sets of $X$, i.e.
                        \begin{equation*}
                            \Op\big(\Lambda(\fsheaf)\big) = \{V_{U,f} \mid U\in\Op X, f\in\fsheaf(U)\}
                        \end{equation*}
                        where $V_{U,f} = (f_x)_{x\in U}$.
                        The local homeomorphism $p\colon\Lambda(\fsheaf)\to X$ is just projection \mbox{$p\colon f_x\mapsto x$} and thus $p^{-1}(x)=\fsheaf_x$ (\textbf{n.b.} hence we call $p^{-1}(x)$ the \textit{stalk} at $x$, and not the fibre!).
                        
                        It turns out that the image under $\Lambda$ of \textit{any} presheaf $\fsheaf\in\Set_X^\pre$ is in fact an étalé space.
                    \item \textit{Equivalence of sheaves and étalé spaces}: the functors $\Gamma$ and $\Lambda$ give an adjunction $(\Gamma\dashv\Lambda)$ between $\Top/X$ and $\Set_X^\pre$.
                        Further, this adjunction descends to an \textbf{equivalence} between $\Etale(X)$ and $\Set_X$ (see \cref{figure:adjunction-diagram-gamma-lambda}).
                \end{itemize}
              

            \subsubsection{Sheafification, inverse image, and compatible germs}
            
                Using the associated étalé space of a presheaf certain constructions become simpler to understand:
                \begin{itemize}
                    \item Sheafification is the composition $\Gamma\Lambda$
                    \item The étalé space of an inverse image is the pullback of the étalé space, i.e. if $\fsheaf\in\Set_X$, $\gsheaf\in\Set_Y$, and $\pi\colon X\to Y$ then \[\Lambda(\pi^{-1}\gsheaf)\cong (X\times_Y\Lambda(\gsheaf)\to X)\] or, equivalently, the following diagram commutes:
                        \begin{equation*}
                            \begin{tikzcd}[row sep=huge, column sep=huge]
                                \Set_X
                                    \ar{d}{\Lambda}
                                &\Set_Y
                                    \ar{l}[swap]{\pi^{-1}}
                                    \ar{d}{\Lambda}\\
                                %%
                                \Etale(X)
                                &\Etale(Y)
                                    \ar{l}[swap]{X\times_Y\Lambda(-)}
                            \end{tikzcd}
                        \end{equation*} 
                \end{itemize}
                
            \subsubsection{Covering spaces and locally-constant sheaves}\label{subsubsection:covering-spaces-locally-constant-sheaves}
            
                \begin{itemize}
                    \item \textit{Covering space over $X$}: a pair $(C,\pi)$ where $C$ is a topological space and $\pi\colon C\to X$ is a (continuous) surjective map such that for all $x\in X$ there exists some open neighbourhood $U_x\subset X$ such that $\pi^{-1}(U_x)=\bigsqcup_{i\in I} V_x^{(i)}$ is a union of disjoint open subsets $V_x^{(i)}\subset C$
                        We write the full subcategory of $\Top/X$ consisting of covering spaces over $X$ as $\Cover(X)$.
                    \item \textit{Locally-constant sheaf}: a sheaf $\fsheaf$ on $X$ such that there exists an open cover $X=\bigcup_{i\in I} U_i$ of $X$ with $\fsheaf|_{U_i}\cong \underline{S^{(i)}}$ for some non-empty set $S^{(i)}$, i.e. `locally, the sheaf is a constant sheaf'.
                    \item \textit{Equivalence of locally-constant sheaves and covering spaces}: the functors $\Gamma$ and $\Lambda$ descend to an \textbf{equivalence} between $\Cover(X)$ and $\Set_X^\lc$ (see \cref{figure:adjunction-diagram-gamma-lambda}).
                \end{itemize}
                
    \section{Affine schemes}
    
        \subsection{Definitions}
        
            \subsubsection{Preliminary definitions}\label{subsubsection:preliminary-definitions-affine-schemes}
        
                \begin{itemize}
                    \item \textit{Spectrum of a ring}: the set $\Spec A=\{[\prid] \mid \prid\ideal_{\mathrm{prime}} A\}$ for some commutative ring $A$ (\textbf{n.b.} we often just write $\prid$ (and $\mathfrak{q}$) to mean a \textbf{prime} ideal, and similarly write $\maid$ to mean a \textbf{maximal} ideal).
                    \item \textit{$\V$ and $\I$}: for $I\ideal A$ define \[\V(I)=\{[\prid]\in\Spec A \mid I\subset\prid\}\subset\Spec A;\] for $S\subset\Spec A$ define \[\I(S)=\bigcap_{[\prid]\in S}\prid\ideal A.\]
                    \item \textit{Zariski topology}: the closed sets of $\Spec A$ are exactly those of the form $\V(I)$ for some $I\ideal A$.
                    \item \textit{Distinguished open set}: given $f\in A$ define \[\D(f)=\{[\prid]\in\Spec A \mid f\not\in\prid\}.\]
                    \item \textit{Elements as functions}: We think of elements $f\in A$ as functions on $\Spec A$ by defining $f([\prid])$ as the image of $f$ in the residue field $\kappa([\prid])$ at $[\prid]$, i.e. \[f([\prid])=f\pmod\prid\in K(A/\prid)\]
                        where $K(R)$ is the \textit{field of fractions} of a ring.
                        We freely use the fact that localising and taking a quotient `commute', i.e. \[\kappa([\prid])=A_\prid/\prid A_\prid = K(A/\prid).\]
                    \item \textit{Structure sheaf}: a sheaf defined on the basis of distinguished opens $\D(f)$ by \[\O_{\Spec A}\big(\D(f)\big)=S^{-1}A\] where $S=\{g\in A \mid \V(g)\subset\V(f)\}=\{g\in A \mid \D(f)\subset\D(g)\}$.
                    \item \textit{Affine scheme}: the set $\Spec A=\{\prid\ideal A\}$ with the \textit{Zariski topology} and \textit{structure sheaf} $\O_{\Spec A}$.
                    \item \textit{Affine $n$-space}: $\A_A^n=\Spec A[x_1,\ldots,x_n]$
                    \item \textit{Morphisms}: a ring homomorphism $\varphi\colon A\to B$ induces a morphism $\Spec B\to\Spec A$ since $\varphi^{-1}(\mathfrak{q})=\prid$, i.e. the preimage of a prime ideal is prime.
                    \item \textit{Specilisation and generisation}: if $x,y\in X$ are points in a topological space such that $x\in\overline{\{y\}}$ then $x$ is a specilisation of $y$, and $y$ is a generisation of $x$.
                    \item \textit{Generic point of a closed subset $K\subset X$}: a point $p\in X$ such that $\overline{\{p\}}=K$.
                    \item \textit{Sheaf from an $A$-module}: for some $A$-module $M$ we define the sheaf $\widetilde{M}$ on the distinguished base by \[\widetilde{M}\big(\D(f)\big)=S^{-1}M=M\otimes_A\O_{\Spec A}\big(\D(f)\big)\] where $S=\{g\in A \mid \V(g)\subset\V(f)\}=\{g\in A \mid \D(f)\subset\D(g)\}$.
                \end{itemize}
        
            \subsubsection{Topological properties}
            
                \begin{itemize}
                    \item \textit{Connected}: $X$ cannot be written as a non-trivial disjoint union of two open sets.
                    \item \textit{Irreducible}: $X$ is non-empty and cannot be written as a non-trivial union of two closed subsets.
                    \item \textit{Quasicompact}: for any open cover $X=\bigcup_{i\in I}U_i$ there exists a finite subcover, i.e. $J\subset I$ with $|J|<\infty$ such that $X=\bigcup_{i\in J}U_i$.
                    \item \textit{Noetherian}: $X$ satisfies the descending chain condition for closed subsets, i.e. any sequence $Z_1\supset Z_2\supset\ldots$ of closed subsets eventually stabilises ($Z_r=Z_{r+1}=\ldots$).
                \end{itemize}
        
        \subsection{Facts}
        
            Here $A$ is a commutative ring, $I$ and $J$ are arbitrary ideals, $S$ is a multiplicative set, $\prid$ is always a prime ideal, etc.
            We are sometimes sloppy with notation just for sake of time (an awful excuse).
            We write $\nilr$ to mean the nilradical ideal of $A$.
        
            \subsubsection{Algebra}
            
                \begin{itemize}
                    \item[(3.4.E)] $\sqrt{\cap_{i=1}^n I_i}=\cap_{i=1}^n\sqrt{I_i}$
                    \item[(3.4.F)] $\sqrt{I}=\cap_{I\subset\prid}\prid$
                    \item $\cap_{\prid\in A}\prid =\nilr$
                    \item If $M$ is an $A$-module and $S\subset A$ is a multiplicative set then $S^{-1}M=M\otimes_A S^{-1}A$.
                \end{itemize}
        
            \subsubsection{Quotients and localisations}
            
                \begin{itemize}
                    \item[(3.2.J), (3.2.K), (3.4.I)] There are inclusion-preserving bijections
                        \begin{align*}
                            \{\prid\ideal A/I\} &\leftrightarrow \{\overline{\prid}\ideal A \mid I\subset\overline{\prid}\};\\
                            \{\prid\ideal S^{-1}A\} &\leftrightarrow \{\overline{\prid}\ideal A \mid \overline{\prid}\cap S=\varnothing\}.
                        \end{align*}
                        These give embeddings
                        \begin{align*}
                            \Spec A/I &\hookrightarrow \Spec A\text{ as a \textit{closed} subset;}\\
                            \Spec A_f &\hookrightarrow \Spec A\text{ as an \textit{open} subset;}\\
                            \Spec S^{-1}A &\hookrightarrow \Spec A\text{ as an arbitrary subset for arbitrary }S\text{.}
                        \end{align*}
                    \item[§3.2.7] $\Spec A/I\subset\Spec A$ `looks like' $\V(I)\subset \Spec A$.
                    \item[§3.2.8] $\Spec A_f=\Spec\big(\{1,f,f^2,\ldots\}^{-1}A\big)$ `looks like' the set of points where $f$ doesn't vanish, i.e. \[\D(f)=\Spec A_f=\Spec A\setminus\V(f).\]
                    \item[§3.2.8] $\Spec A_\prid=\Spec\big((A\setminus\prid)^{-1}A\big)$ `looks like' a shred of $\Spec A$ near the subspace corresponding to $[\prid]$ (i.e. $\V(\prid)$).
                \end{itemize}
            
            \subsubsection{Nilpotents}
            
                \begin{itemize}
                    \item[(3.4.5)] If $I\subset\nilr$ is a radical of nilpotents then $\Spec A/I\to\Spec A$ is a homeomorphism.
                \end{itemize}
            
            \subsubsection{Distinguished opens}
            
                \begin{itemize}
                    \item[(3.5.A)] $\Spec A\setminus\V(I)=\bigcup_{f\in I}\D(f)$; in particular the distinguished open sets for a basis for the Zariski topology.
                    \item[(3.5.D)] $\D(f)\cap\D(g)=\D(fg)$.
                    \item[(3.5.E)] $\D(f)\subset\D(g)$ iff $f\in\sqrt{(g)}$ iff $g^{-1}\in A_f$.
                    \item[(3.5.B)] $\bigcup_{i\in I}\D(f_i)=\Spec A$ iff $(f_i)_{i\in I}= A$ iff there exists $J\subset I$ with $|J|<\infty$ such that $A=\sum_{i\in J}a_if_i$ for some $a_i\in A$.
                    \item[(3.5.F)] $\D(f)=\varnothing$ iff $f\in\nilr$.
                \end{itemize}
                
            \subsubsection{$\V$ and $\I$}\label{subsubsection:v-and-i}
            
                \begin{itemize}
                    \item[(3.6.L)] $[\mathfrak{q}]$ is a specilisation of $[\prid]$ iff $\prid\subset\mathfrak{q}$, thus in particular $\V(\prid)=\overline{\{\prid\}}$.
                    \item $\V(I)\cup\V(J)=\V(IJ)$.
                    \item $\V(I)\cap\V(J)=\V(I+J)$.
                    \item[(3.6.I), (3.7.1), (3.7.E)] $\V$ and $\I$ give an inclusion-reversing bijection
                        \begin{align*}
                            \{\text{radical ideals of }A\} &\leftrightarrow \{\text{closed subsets of }\Spec A\};\\
                            \{\text{prime ideals of }A\} &\leftrightarrow \{\text{irreducible closed subsets of }\Spec A\};\\
                            \{\text{maximal ideals of }A\} &\leftrightarrow \{\text{closed points of }\Spec A\}.
                        \end{align*}
                    \item[(3.7.E)] Every irreducible closed subset of $\Spec A$ has exactly one generic point.
                    \item The maximal ideals of $k[x_1,\ldots,x_n]$ when $k$ is algebraically closed are exactly $(x_1-a_1,\ldots,x_n-a_n)$ for $(a_1,\ldots,a_n)\in A^n$.
                \end{itemize}
            
            \subsubsection{Structure sheaf}
            
                \begin{itemize}
                    \item[(4.1.A)] The natural map $A_f\to\O_{\Spec A}\big(\D(f)\big)$ is an isomorphism.
                    \item[(4.3.2)] The global sections of the structure sheaf of an affine scheme give the associated ring, i.e. if $(X,\O_X)\cong(\Spec A,\O_{\Spec A})$ as ringed spaces (see \cref{subsubsection:schemes-preliminary-definitions}) then \[\Gamma(X,\O_X) = \Gamma\big(\D(1),\O_{\Spec A}\big) = A.\]
                    \item[(4.3.F)] The stalk $\O_{X,p}$ of the structure sheaf $\O_X$ at a point $p=[\prid]$ is the local ring $A_\prid$. In particular, affine schemes (and thus schemes) are locally ringed spaces.
                \end{itemize}
                
            \subsubsection{Topological properties}
            
                \begin{itemize}
                    \item[(3.6.3)] $\Spec A$ is \textbf{not} connected if and only if $A\cong A_1\times A_2$ for some non-zero rings $A_1$, $A_2$.
                    \item[(3.6.B)] If $Z\subset X$ is irreducible (with the subspace topology) then so too is $\overline{Z}$.
                    \item[(3.6.C)] If $A$ is an integral domain then $\Spec A$ is irreducible.
                    \item[(3.6.D)] An irreducible topological space is connected (but the converse is not necessarily true).
                    \item[(3.6.G)] $\Spec A$ is quasicompact for all $A$ (\textbf{n.b.} but in general it can still have non-quasicompact open subsets, unless $\Spec A$ is Noetherian).
                    \item[(3.6.H)] Every closed subset of a quasicompact space is quasicompact.
                    \item[(3.6.I)] Closed points of $\Spec A$ (i.e. points $p$ such that $\{p\}$ is closed) correspond to maximal ideals (see \cref{subsubsection:v-and-i}).
                    \item[(3.6.J)] If $k$ is a field and $A$ is a finitely-generated $k$-algebra then the closed points of $\Spec A$ are dense.
                    \item[(3.6.O)] Connected components of a topological space are closed (if the space is Noetherian then they are open too).
                    \item[(3.6.Q)] Every connected component $C$ of a topological space $X$ is the union of irreducible components of $X$.
                    \item[(3.6.Q)] If $S\subset X$ is both open and closed then it is the union of some of the connected components of $X$ (if $X$ is Noetherian then the converse holds too).
                    \item[(3.6.T)] If $A$ is Noetherian then $\Spec A$ is Noetherian.
                \end{itemize}

    \section{Schemes}
    
        \subsection{Definitions}
        
            \subsubsection{Preliminary definitions}\label{subsubsection:schemes-preliminary-definitions}
            
                \begin{itemize}
                    \item \textit{Isomorphism of ringed spaces}: $\pi\colon(X,\O_X)\congto(Y,\O_Y)$ is the data of a homeomorphism $\pi:X\to Y$ and an isomorphism of sheaves $\O_Y\congto\pi_*\O_X$ (or equivalently $\pi^{-1}\O_Y\congto\O_X$).
                    \item \textit{Affine scheme}: \textbf{redefined} as any ringed space isomorphic to $(\Spec A,\O_{\Spec A})$ for some $A$.
                    \item \textit{Scheme}: a ringed space $(X,\O_X)$ such that every point $x\in X$ has an open neighbourhood $U$ with $(U,\O_X|_U)$ isomorphic to an affine scheme (\textbf{n.b.} although we don't specify a topology on $X$, the fact that we require it to consist locally of affine schemes means that we still call its topology the Zariski topology – see (4.3.D)).
                    \item \textit{Isomorphism of schemes}: this is just an isomorphism of ringed spaces.
                    \item \textit{Open subscheme}: the ringed space $(U,\O_X|_U)$ for any open subset $U\subset X$ of a scheme.
                    \item \textit{Closed subscheme}: the ringed space $(Z,i^{-1}(\O_X/\mathcal{I}))$ for any closed subset $i\colon Z\hookrightarrow X$ of a scheme and $\mathcal{I}\subset\O_X$ an ideal sheaf.
                \end{itemize}
                
            \subsubsection{Projective schemes}
            
                Unless otherwise stated, $S_\bullet$ is $\Z_{\geqslant0}$-graded.
            
                \begin{itemize}
                    \item \textit{Graded ring over $A$}: a graded ring $S_\bullet$ with $S_0=A$; the irrelevant ideal is $S_+=\bigoplus_{d\geqslant1}S_d$; $\S_\bullet$ is \textit{finitely generated over $A$} if $S_+$ is a finitely-generated ideal; $S_\bullet$ is \textit{generated in degree one} if it is generated by $S_1$ as an $A$-algebra.
                    \item \textit{$\Proj$ construction}: we define the scheme $\Proj S_\bullet$ as follows.
                        \begin{itemize}
                            \item As a set, \[\Proj S_\bullet = \big\{[\prid]\ideal S_\bullet \mid \prid\text{ is homogeneous}, S_+\not\in\prid\big\};\]
                            \item As a topological space (with the \textit{Zariski} topology), the closed sets are the $\V(I)$ for \textbf{homogeneous} ideals $I\ideal S_\bullet$ where \[\V(I) = \{[\prid]\in\Proj S_\bullet \mid I\subset\prid\}\]
                                and we define $\D(f)=\Proj S_\bullet\setminus\V(f)$ as before;
                            \item As a scheme, the structure sheaf $\O_{\Proj S_\bullet}$ is defined by gluing together sheaves on the open subschemes $\D(f)$, where we define \[\O_{\Proj S_\bullet}|_{\D(f)} = \O_{\Spec((S_\bullet)_f)_0}\]
                                where $((S_\bullet)_f)_0$ is the degree-zero part of the localisation of $S_\bullet$ at $\{1,f,f^2,\ldots\}$.
                        \end{itemize}
                    \item \textit{Projective $n$-space}: $\P_A^n=\Proj A[x_0,\ldots,x_n]$.
                    \item \textit{The sheaf associated to a graded module}: (recall the definition of $\widetilde{M}$ from \cref{subsubsection:preliminary-definitions-affine-schemes}). Let $M$ be a graded module over $S_\bullet$ and define the sheaf $\widetilde{M}$ by \[\widetilde{M}\big(\D(f)\big) = \widetilde{(M_f)_0}.\]
                    \item \textit{Serre twisting sheaf}: (\textbf{n.b.} this comes much later in Vakil: §14, §15). When $S_\bullet$ is \textbf{generated in degree one}, for $n\in\Z_{\geqslant0}$ define $M(n)=S_\bullet[n]$ i.e. $M_d=S_{d+n}$. Then write \[\O(n) = \widetilde{M(n)} = \widetilde{S_\bullet[n]}.\]
                        For $-n\in\Z_{\geqslant0}$ (i.e. $n<0$) define \[\O(-n) = \O(n)^\vee.\]
                \end{itemize}
                
        \subsection{Facts}
        
            \subsubsection{Schemes and ringed spaces}
            
                \begin{itemize}
                    \item[(4.3.B)] If $f\in A$ then $\D(f)\leftrightarrow\Spec A_f$ induces a natural isomorphism of ringed spaces \[\big(\D(f),\O_{\Spec A}|_{\D(f)}\big) \cong (\Spec A_f,\O_{\Spec A_f})\] i.e. they are isomorphic as affine schemes.
                    \item[(4.3.D)] The affine open subsets of a scheme form a basis for the (Zariski) topology on $X$.
                    \item[(4.3.G)] If $f$ is a function on a ringed space $X$ (i.e. $f\in\Gamma(X,\O_X)$) then \[\{x\in X \mid f_x\text{ is invertible}\}\text{ is open in }X.\]
                        If $X$ is \textbf{locally} ringed then \[\{x\in X \mid f(x)=0\}\text{ is closed in }X.\]
                        Thus if $f$ vanishes nowhere then it is invertible.
                \end{itemize}
                
            \subsubsection{Algebra}
            
                \begin{itemize}
                    \item [4.5.D] If $S_\bullet$ is a graded ring over $A$ then it is finitely-generated (as a graded ring over $A$) iff it is finitely-generated as a graded $A$-algebra (i.e. generated over $A=S_0$ by a finite number of homogeneous elements of positive degree).
                \end{itemize}

\end{document}
