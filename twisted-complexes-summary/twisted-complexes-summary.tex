\documentclass[11pt,notitlepage]{article}

%%% PREAMBLE %%%

    %%% Tools/essentials %%%

    \usepackage[unicode]{hyperref}
    \usepackage{amsmath,amssymb,amsthm}
    \usepackage[utf8]{inputenc}
    \usepackage[capitalize]{cleveref}
    \usepackage{embedfile}
    \usepackage{mathtools}

    %%% Bibliography %%%

    \usepackage[backend=biber,style=alphabetic]{biblatex}
    \bibliography{twisted-complexes-summary}

    %%% Typography %%%

    \usepackage{charter}
    \usepackage{enumerate}
    \usepackage{url}
    \usepackage[perpage]{footmisc}
    \usepackage{mathrsfs}

    %%% Other %%%

    \usepackage{tikz-cd}

    %%% Titles, headers, and footers %%%

    \usepackage{titlesec}

    \titlespacing*{\section}
    {0pt}{10.5ex plus 1ex minus .2ex}{5.3ex plus .2ex}
    \titlespacing*{\subsection}
    {0pt}{5.5ex plus 1ex minus .2ex}{4.3ex plus .2ex}
    \titlespacing*{\subsubsection}
    {0pt}{5.5ex plus 1ex minus .2ex}{2.3ex plus .2ex}

    \titleformat*{\section}{\centering\huge\sffamily}
    \titleformat*{\subsection}{\Large\sffamily}
    \titleformat*{\subsubsection}{\large\sffamily}

    \usepackage[left=1.5in, right=1.5in, footskip=1in, bottom=1.5in]{geometry}

    %%% Theorem environments %%%

    \numberwithin{equation}{subsection}

    \usepackage{thmtools}

    \declaretheoremstyle[
        spacebelow=2\topsep,
        spaceabove=2\topsep,
        headfont=\normalfont\bfseries,
        bodyfont=\itshape,
        postheadspace=\newline,
        qed=${\lrcorner}$,
        headpunct={},
        notebraces={[}{]}
    ]{breakit}

    \declaretheoremstyle[
        spacebelow=2\topsep,
        spaceabove=2\topsep,
        headfont=\normalfont\bfseries,
        bodyfont=\normalfont,
        postheadspace=\newline,
        qed=${\lrcorner}$,
        headpunct={},
        notebraces={[}{]}
    ]{breakup}

    \declaretheorem[numberlike=equation,style=breakit]{theorem}
    \declaretheorem[numberlike=equation,style=breakit]{lemma}
    \declaretheorem[numberlike=equation,style=breakit]{corollary}

    \declaretheorem[numberlike=equation,style=breakup]{definition}
    \declaretheorem[numberlike=equation,style=breakup]{example}
    \declaretheorem[numberlike=equation,style=breakup]{note}

    %%% Shortcuts %%%

    \DeclareMathOperator{\End}{End}
    \DeclareMathOperator{\Hom}{Hom}
    \DeclareMathOperator{\GL}{GL}
    \DeclareMathOperator{\id}{id}
    \DeclareMathOperator{\Tot}{Tot}

    \newcommand{\congto}{\xrightarrow{\sim}}

    \renewcommand{\H}{\operatorname{H}}
    \renewcommand{\d}{\mathrm{d}}
    \renewcommand{\C}{\mathscr{C}}

%%% DOCUMENT %%%

\embedfile{\jobname.tex}

\begin{document}

    %%% Pre-content %%%
        
        %%% Title %%%

            \author{%
                Timothy Hosgood\footnote{%
                    Université d'Aix-Marseille;
                    \href{mailto:timhosgood@gmail.com}{\texttt{timhosgood@gmail.com}}%
                    }%
            }%
            \title{Twisted complexes and twisting cochains\\\Large\textit{Motivation, history, and summary}}
            \maketitle

        %% Abstract %%%

            \begin{abstract}
                The terms \textit{twisting cochain} and \textit{twisted complex} appear in many different places in mathematical literature, with many cross-references between occurrences, but not always with detailed explanations of how these concepts are linked.
                This paper aims to serve as an introductory guide and literature review to the notion of twisting cochains (and twisted complexes) and their use in algebraic geometry and algebraic topology.
            \end{abstract}

        %%% Table of contents %%%

            \vspace{-3em}
            \tableofcontents

            \subsubsection*{Thanks}
                {\small The author is, as always, thankful for the help, advice, and motivation from Julien Grivaux and Damien Calaque (his two supervisors).}
                {\small In particular, D. Calaque explained very clearly how the twisting cochains of Toledo and Tong can be viewed as explicit examples of the twisted complexes of Bondal and Kapranov; and J. Grivaux suggested this topic and answered numerous questions on pretty much everything even slightly related to the subject.}

    %%% Content %%%

        \section{Introduction}

            \subsection{Subject}

                Twisting cochains and twisted complexes arise naturally in many different settings.
                They seem to become a very useful tool when studying complex-analytic manifolds, since they can be used as a substitute for the global resolutions of coherent sheaves that we have in the algebraic setting.

            \subsection{Historical overview}

                No historical account is ever complete, and this is especially true with an overview of the literature concerning a single subject in mathematics; it is impossible to discuss an idea from every angle and to find all of the places where it has turned up in some form or another.

                In particular, we don't really mention the `first' reference to twisted cochains: \cite{Brown:1959ia}; we also don't follow what happened to the subject when it branched off into differential homological algebra (namely \cite{Moore:70dh}), even though this also predates all of the material that we \textit{do} cover.\footnote{For a good summary of the subject from a differential and lie-algebraic viewpoint, see \cite{Stasheff:2009cc}.}
                Our focus is really split into two parts: firstly, the development of twisting cochains by Toledo and Tong (and O'Brian) in \cite{Toledo:1976gy,Toledo:1978tq,OBrian:1981vs} using Čech cohomology and explicit methods; secondly, the development of twisted complexes (from \cite{Bondal:1991un}\footnote{In some papers this is cited as \textit{Framed triangulated categories} instead of \textit{Enhanced triangulated categories}, but this is just an artefact of translation from the original paper (which is in Russian).}) and the application of the language of DG-categories in \cite{Block:2015vk,Wei:2016tv,Wei:2016ip}.
                We discuss only briefly some of the generalisations to $A_\infty$-categories, such as \cite{Faonte:2015vc}.

            \subsection{Purpose}

                This paper is largely just notes on, and summaries of, important papers in the literature.
                It is intended as a kick-start guide for anybody wishing to work with twisted complexes.

        \section{Holomorphic twisting cochains}\label{sec:holomorphic-twisting-cochains}

            We discuss and summarise (certain sections of) two seminal papers by Toledo and Tong.
            Each provides a slightly different viewpoint on why twisting cochains are interesting and useful, and \cite{OBrian:1981vs} is a very good demonstration of some of their applications (and also deals with the definitions in a slightly more abstract manner, using spectral sequences to discuss various properties).

            \subsection{Vector bundles and graded vector spaces}\label{ssub:vector-bundles-and-graded-vector-spaces}

                \emph{We have decided to start with \cite{Toledo:1978tq} and not the earlier \cite{Toledo:1976gy} because, apart from a slight reversal of definitions\footnote{That is, here we start with some graded vector space and define a complex structure on it by defining a differential, whereas what often happens is we \textit{already have} a collection of complexes and wish to use the differential that comes with them. There is no real mathematical difference, but conceptually the approaches are slightly different.}, this approach closely resembles what we wish to define in \cref{ssub:twisting-resolutions}.}

                \bigskip

                Let $V^\bullet=\{V^\bullet_\alpha\}$ be a collection of graded finite-dimensional $\mathbb{C}$-vector spaces\footnote{That is, $V_\alpha=\bigoplus_{i\in\mathbb{N}}V_{i,\alpha}$.} and write $L^\bullet(V)=\End^\bullet(V)$, where $\End^q(V)$ consists of degree-$q$ endomorphisms.
                If we let $X$ be some paracompact complex-analytic $N$-manifold then we define $\underline{V}^\bullet = \mathcal{O}_X\otimes_\mathbb{C}V^\bullet$ and $\underline{L}^\bullet(V) = \mathcal{O}_X\otimes_\mathbb{C}\End^\bullet(V)$.
                Then we let $\underline{U}=\{U_\alpha\}$ be a `sufficiently-nice\footnote{In particular, $\underline{U}$ is Stein, locally compact, and, if we are working with some locally-free sheaf $\mathcal{F}$ of rank $r$, assumed to be such that $\mathcal{F}|_{U_\alpha}\cong\mathcal{O}_X^r$.}' open cover of $X$, where the $\alpha$ correspond to the $\alpha$ for $V^\bullet$ (i.e. there is a $V^\bullet_\alpha$ for each $U_\alpha$).

                Finally, for any complex of vector bundles $K^\bullet$, we define a Čech-style complex $\hat{\C}^\bullet(\underline{U},K^\circ)$ by letting an element of $\hat{\C}^p(\underline{U},K^q)$ be a Čech $p$-cochain $c$ whose value $c_{\alpha_0\ldots\alpha_p}$ on the $p$-simplex $U_{\alpha_0\ldots\alpha_p}$ lies in $K^q(U_{\alpha_0})|_{\alpha_0\ldots\alpha_p}$.
                The Čech differential on this complex is actually a \textit{deleted} Čech differential, written $\hat{\delta}$ instead of the usual $\check{\delta}$, i.e. it is the usual alternating sum of the cochain on $\alpha_0\ldots\hat{\alpha}_i\ldots\alpha_p$ but where we start from $i=1$, \emph{not} $i=0$, and go up to $i=p+1$.
                This is natural, because we have defined our cochains to take values in $K(U_{\alpha_0})$, and something of the form $c_{\alpha_1\ldots\alpha_{p+1}}$ does \textit{not} live here.
                This double complex also has a natural multiplication structure\footnote{In fact, it often has an algebra structure, but this isn't too important for us here.} given by $(c^p\cdot d^q)_{\alpha_0\ldots\alpha_{p+q}} = c^p_{\alpha_0\ldots\alpha_p}d^q_{\alpha_p\ldots\alpha_{p+q}}$.
                This structure plays nicely with all the differentials involved.

                \bigskip

                Now, a holomorphic vector bundle on $X$ is defined by a holomorphic $1$-cocycle with coefficients in $\GL(n,\mathbb{C})$, i.e. the transition maps on overlaps.
                We generalise this slightly by forgetting the grading on $V$ and giving a $1$-cochain $\{\mathrm{a}_{\alpha\beta}\}_{\alpha,\beta} \in \hat{\C}^1(\underline{U},\underline{L}(V))$ satisfying the cocycle condition
                \begin{equation}\label{eq:cocycle-condition}
                    \mathrm{a}_{\alpha\beta}\cdot\mathrm{a}_{\beta\gamma}=\mathrm{a}_{\alpha\gamma}
                \end{equation}
                and \textit{also} satisfying $\mathrm{a}_{\alpha\alpha}=\id$ (to ensure that our cochains take values in $\GL(V)$).

                But note that \cref{eq:cocycle-condition} is equivalent to
                \begin{equation}
                    \hat{\delta}\mathrm{a}+\mathrm{a}\cdot\mathrm{a},
                \end{equation}
                and \textit{this} is the equation that we are going to generalise.
                Since we forgot the grading on $V^\bullet$, we can think of $\mathrm{a}$ as $\mathrm{a}^{1,0}$, i.e. a $1$-cocycle of degree $0$.
                This then leads us to the following definition.

                \begin{definition}[Twisting cochain]\label{def:twisting-cochain}
                    A \textit{twisting cochain} over $V^\bullet$ is an element
                    \[\mathrm{a}=\sum_{k\geqslant0}\mathrm{a}^{k,1-k}\]
                    of total degree $1$ in the double complex\footnote{We don't have any negative values of $k$ because by our definition the Čech degree of any element must be at least zero.}, where $\mathrm{a}^{k,1-k}\in\hat{\C}^k(\underline{U},\underline{L}^{1-k}(V))$, such that
                    \begin{align*}
                        \hat{\delta}\mathrm{a}+\mathrm{a}\cdot\mathrm{a}&=0\\
                        \mathrm{a}^{1,0}_{\alpha\alpha}&=\id.\qedhere
                    \end{align*}
                \end{definition}

                \begin{note}
                    In the double complex, multiplication is \textit{not} simply done component-wise.
                    Instead, we take all possible combinations, i.e. $(\mathrm{a}\cdot\mathrm{b})^{k,k'}=\sum_{m+n=k,m'+n'=k'}\mathrm{a}^{m,m'}\cdot\mathrm{b}^{n,n'}$.
                \end{note}

                By looking at the equations in \cref{def:twisting-cochain} seperately, in each degree, we can see why this is a `good' generalisation.
                \begin{itemize}
                    \item For $k=0$ we simply get that $\mathrm{a}^{0,1}_{\alpha}\mathrm{a}^{0,1}_{\alpha}=0$, that is, $\mathrm{a}^{0,1}_{\alpha}$ is an \textit{$\mathcal{O}_X$-linear differential on the complex $\underline{V}^\bullet|_{U_\alpha}$}.
                    \item For $k=1$ we get that $\mathrm{a}^{0,1}_{\alpha}\mathrm{a}^{1,0}_{\alpha\beta} = \mathrm{a}^{1,0}_{\alpha\beta}\mathrm{a}^{0,1}_{\beta}$, that is, over $U_{\alpha\beta}$ the map
                        \[\mathrm{a}^{1,0}_{\alpha\beta}\colon (\underline{V}^\bullet|_{U_\alpha\beta},\mathrm{a}^{0,1}_{\beta})\to (\underline{V}^\bullet|_{U_\alpha\beta},\mathrm{a}^{0,1}_{\alpha})\]
                        is a \textit{chain map of complexes}.
                    \item For $k=2$, in the case of degenerate simplices $(\alpha,\beta,\alpha)$ and $(\beta,\alpha,\beta)$, we see that $\mathrm{a}^{1,0}_{\alpha\beta}$ and $\mathrm{a}^{1,0}_{\beta\alpha}$ are \textit{chain homotopic inverses}, that is, the chain map $\mathrm{a}^{1,0}_{\alpha\beta}$ is a \textit{quasi-isomorphism}.
                        For general simplices, we get that $\mathrm{a}^{1,0}_{\alpha\gamma}$ and $\mathrm{a}^{1,0}_{\alpha\beta}\mathrm{a}^{1,0}_{\beta\gamma}$ are chain homotopic.
                    \item For $k\geqslant3$ we get `higher-order homotopy gluings'.\footnote{Whatever this might mean, formally.}
                \end{itemize}

                In an ideal world we would hope to be able to glue the complexes $V^\bullet|_{U_\alpha}$ and $V^\bullet|_{U_\beta}$ in order to obtain some global object, but if this isn't possible then the next best thing to hope for is a twisting cochain, which lets us glue things `up to homology'.
                To make this precise, the data of a twisting cochain $\mathrm{a}$ gives us
                \begin{itemize}
                    \item $\mathcal{H}^\bullet_\alpha=\H^\bullet(\underline{V}^\bullet|_{U_\alpha})$, which is a complex of locally-free sheaves;
                    \item over each intersection $U_{\alpha\beta}$ we get an isomorphism $\H(\mathrm{a}^{1,0}_{\alpha\beta})\colon\mathcal{H}^\bullet_\beta|_{U_{\alpha\beta}}\congto\mathcal{H}^\bullet_\alpha|_{U_{\alpha\beta}}$ \textit{such that} over $U_{\alpha\beta\gamma}$ we have the isomorphism $\H(\mathrm{a}^{1,0}_{\alpha\gamma})\cong\H(\mathrm{a}^{1,0}_{\alpha\beta})\H(\mathrm{a}^{1,0}_{\beta\gamma})$;
                    \item some higher order gluing conditions.
                \end{itemize}

                The above data lets us define a complex of coherent sheaves $\mathcal{H}^\bullet(\mathrm{a})$.
                If we further have that $V^i=0$ for $i>0$, $\mathcal{H}^0(\mathrm{a})\cong\mathcal{F}$ for some coherent sheaf $\mathcal{F}$, and $\mathcal{H}^i(\mathrm{a})=0$ otherwise, then we can think of the collection $\{(\underline{V}^\bullet|_{U_\alpha},\mathrm{a}^{0,1}_\alpha)\}_{\alpha}$ as a \textit{local resolution of $\mathcal{F}$ by locally-free sheaves}.

                \begin{note}
                    An important idea is that we can build a twisting cochain inductively by using some classical results in homological algebra to lift certain identity maps, and then constructing the higher $\mathrm{a}^{k,1-k}$ to satisfy the equation $\hat{\delta}\mathrm{a}+\mathrm{a}\cdot\mathrm{a}=0$.
                    This is covered in detail in \cite[p.~230]{OBrian:1981vs} and \cite[Lemma~8.13]{Toledo:1976gy}.
                \end{note}

                \bigskip

                Finally, we talk about the total complex structure that we get from a twisting cochain.
                It is not too hard (but is certainly tiring) to show that the conditions for $\mathrm{a}$ to be a twisting cochain are equivalent to the conditions for the morphism $D_\mathrm{a}\colon c\mapsto\hat{\delta}c+\mathrm{a}\cdot c$ to be a (degree-$1$) differential on $\Tot^\bullet\hat{\C}^p(\underline{U},\underline{V}^q)$, i.e. $D_\mathrm{a}^2=0$.

                Why would we want to consider a differential of this form?
                Well, conceptually, this looks like a first order perturbation of our deleted Čech differential: this is the point of view that we consider in \cref{ssub:connections}.
                As a more concrete reason though, we can work through two specific examples.

                \begin{enumerate}
                    \item (Vector bundle: $\underline{V}$ ungraded and $\mathrm{a}=\mathrm{a}^{1,0}$).
                        Returning to the example that we started with, we see that here
                        \[(D_\mathrm{a}c^p)_{\alpha_0\ldots\alpha_{p+1}} = \mathrm{a}^{1,0}_{\alpha_0\alpha_1}c^p_{\alpha_1\ldots\alpha_{p+1}} + \sum_{i=1}^{p+1}(-1)^ic^p_{\alpha_0\ldots\hat{\alpha}_i\ldots\alpha_{p+1}}.\]
                        But recall that $\mathrm{a}^{1,0}_{\alpha_0\alpha_1}$ is a quasi-isomorphism, and so gives us a way of thinking of $c^p_{\alpha_1\ldots\alpha_{p+1}}$ as living in $\underline{L}(U_{\alpha_0})|_{\alpha_0\ldots\alpha_{p+1}}$, i.e. it fixes the problem we had with not being able to define the `real' Čech differential on our complex and having to use the deleted one: $D_\mathrm{a}$ `is' $\check{\delta}$.
                    \item (Complex of vector bundles: $\underline{V}^\bullet$ graded, $\mathrm{a}=\mathrm{a}^{0,1}+\mathrm{a}^{1,0}$).
                        Here we consider the case when we have a \textit{real} resolution (i.e. no higher-order gluing conditions) of some (coherent) sheaf by locally-free sheaves.
                        Recalling that $\mathrm{a}^{0,1}$ is the differential (written $\d_V$) on $V^\bullet$, we see that
                        \[D_\mathrm{a}c^{p,\bullet} = \check{\delta}c^{p,\bullet} + (-1)^p\d_V c^{p,\bullet}.\]
                        That is, we recover the usual total Čech bicomplex with differential $\check{\delta}\pm\d_V$.
                \end{enumerate}

                Interestingly enough, we can actually start from this total differential and define twisting cochains from there.
                This is the approach that we study in \cref{ssub:connections}.

            \subsection{Connections}\label{ssub:connections}

                \emph{Twisting cochains originally arose in a more general setting, for general bigraded algebras, in \cite[§8]{Toledo:1976gy}, and were defined starting from the total differential $D_\mathrm{a}$ that we defined at the end of \cref{ssub:twisting-resolutions}.}
                \emph{The setting is as follows.}

                \bigskip

                Let $A=\bigoplus_{p,q}A^{p,q}$ be a bigraded $\mathbb{C}$-algebra and $M=\bigoplus_{p,q}M^{p,q}$ a bigraded $A$-module (and thus also a bigraded $\mathbb{C}$-module).
                We assume that $A$ and $M$ are both zero for $p<0$ and only non-zero for finitely-many $q$.
                Further, we take $M$ to be a faithful $A$-module\footnote{Another way of saying this is that we can identify $A$ with a subalgebra of $\End_\mathbb{C}(M)$.}: if $a\cdot m=0$ for all $m$ then $a=0$.

                Now take two $\mathbb{C}$-linear maps of bidegree $(1,1)$, say $D=D'+D''$ on $A$ and $\nabla=\nabla'+\nabla''$ on $M$, where $D'$ and $\nabla'$ are of bidegree $(1,0)$ and $D''$ and $\nabla''$ are of bidegree $(0,1)$.
                Further, take both $D$ and $\nabla$ to be derivations:
                \begin{align*}
                    D(a\cdot b) &= (Da)\cdot b + (-1)^{|a|}a\cdot(Db)\\
                    \nabla(a\cdot m) &= (Da)\cdot m + (-1)^{|a|}a\cdot(\nabla m)
                \end{align*}
                where $|a|=p+q$ for $a\in A^{p,q}$.
                We assume also that $D''$ and $\nabla''$ are differentials (i.e. square to zero), but we place \textit{no} such assumption on $D'$ and $\nabla'$.
                In fact, in general, $(\nabla')^2$ will be a non-zero endomorphism of bidegree $(2,0)$, but we \textit{do} assume that there exists some $k\in A^{2,0}$ such that $(\nabla')^2m=k\cdot m$ for all $m$.
                Our final assumption is that the following anti-commutativity relations are satisfied:
                \begin{align*}
                    \nabla'\nabla'' &= -\nabla''\nabla\\
                    D'D'' &= - D''D'.
                \end{align*}

                \bigskip

                A situation that will often arise when we have maps $D$ and $\nabla$ satisfying the properties above is that we want to modify $\nabla$ by some element of $A$ to obtain a true differential (we justify this statement shortly).
                Explicitly, we want to find some $\mathrm{a}=\sum_{i\geqslant0}\mathrm{a}^{i,1-i}$ such that $\nabla_\mathrm{a}^2=0$, where we define $\nabla_\mathrm{a}m = \nabla m+a\cdot m$ satisfies .
                Note that
                \[
                    \nabla_\mathrm{a}^2m = \nabla^2m+\mathrm{a}\cdot\nabla m+\nabla(\mathrm{a}\cdot m)+\mathrm{a}\cdot\mathrm{a}\cdot m.
                \]
                Using the derivation property and the definition of $k$ we see that
                \[
                    \nabla_\mathrm{a}^2m = (D\mathrm{a}+\mathrm{a}\cdot\mathrm{a}+k)\cdot m
                \]
                for all $m$, and thus, since $M$ is faithful, $\nabla_\mathrm{a}^2=0$ if and only if $D\mathrm{a}+\mathrm{a}\cdot\mathrm{a}+k=0$.
                This leads to the following definition.
                \begin{definition}[Twisting cochain]
                    A \textit{twisting cochain} for $(M,\nabla)$ is an element $\mathrm{a}=\sum_{i\geqslant0}\mathrm{a}^{i,1-i}\in A$ such that
                    \[
                        D\mathrm{a}+\mathrm{a}\cdot\mathrm{a}=-k.
                    \]
                    The \textit{twisted complex} associated to $\mathrm{a}$ is the complex $(M,\nabla_\mathrm{a})$.
                \end{definition}

                As for the justification for this definition, we look to differential geometry: if we think of $\nabla$ as a connection then $k$ corresponds to the curvature, and the definition $\nabla_\mathrm{a}=\nabla+\mathrm{a}$ looks like the local definition of a connection, i.e. it is just $\nabla$ modified by an operator \textit{of order zero}, and the fact that $\mathrm{a}$ is a twisting cochain tell us that this new connection $\nabla_\mathrm{a}$ is \textit{flat}.
                Also note that, when $k=\mathrm{a}=0$, the twisted complex is simply our original bicomplex.

                \bigskip

                An important thing to note is that we have the following progression of relations:
                \begin{enumerate}
                    \item $\nabla^2m = k\cdot m$ for all $m$;
                    \item $(D^2a)\cdot m = \nabla^2(a\cdot m)-a\cdot\nabla^2m = (k\dot a-a\cdot k)\cdot m$ for all $m$, by using the derivation assumption;
                    \item $D^2a=k\cdot a-a\cdot k$, from the above, since $M$ is faithful;
                    \item $k\cdot\nabla m = Dk\cdot m+k\cdot\nabla m$ for all $m$, by writing $\nabla^3m$ in two ways (using the anti-commutativity relations and by using the derivation assumption);
                    \item $Dk=0$, from the above, since $M$ is faithful.
                \end{enumerate}
                Although we do not make use of these here, we state them because they are vital to the fact that, under sufficiently nice conditions\footnote{Roughly speaking, we want $A$ to be $D''$-acyclic.}, we can always inductively construct a twisting cochain (see \cite[Lemma~8.13]{Toledo:1976gy}).

                \bigskip

                Finally, we briefly\footnote{\emph{Very} briefly, because the story of Maurer-Cartan forms and the Chevally-Eilenberg algebra and deformations and all this is something that we really deserves its own set of notes.} mention the \textit{Maurer-Cartan} equations.
                If $(\mathfrak{g},[-,-],\partial)$ is a dg-Lie algebra\footnote{This can be phrased in terms of DG-categories, but is also equivalently a Lie algebra along with a differential that acts as a graded derivation with respect to the Lie bracket.} with homological grading (i.e. $\partial$ is a differential of degree $-1$) then a \textit{Maurer-Cartan element} is some $\mathrm{a}\in\mathfrak{g}$ of degree $-1$ such that
                \[
                    \partial\mathrm{a}+\frac12[\mathrm{a},\mathrm{a}]=0.
                \]

            \subsection{Twisting resolutions}\label{ssub:twisting-resolutions}

                One important application of the holomorphic twisting cochains of Toledo and Tong is in the idea of a \emph{twisting resolution of a sheaf}.
                
                \bigskip

                Let $X$ be a paracompact complex $N$-manifold, with sheaf of holomorphic functions $\mathcal{O}_X$, and let $\underline{U}=\{U_\alpha\}$ be a sufficiently-nice\footnote{Here, this really just means locally finite.} open cover.
                Suppose that over each $U_\alpha$ we have a finite-length complex $(E_\alpha^\bullet,\d_\alpha)$ of locally-free $\mathcal{O}_{U_\alpha}$-modules.

                Define 
                \[
                    \End^q(E)|_{U_{\alpha_0\ldots\alpha_p}} = \Big\{\big(f^i\colon E_{\alpha_p}^i|_{U_{\alpha_0\ldots\alpha_p}}\to E_{\alpha_0}^{i+q}|_{U_{\alpha_0\ldots\alpha_p}}\big)_{i\in\mathbb{Z}} \,\Big| \d_{\alpha_p}\circ\,f^i = f^{i+1}\circ\,\d_{\alpha_0}\Big\}.
                \]

                \begin{note}
                    There are two important differences here when compared to \cref{ssub:vector-bundles-and-graded-vector-spaces}:
                    \begin{enumerate}
                        \item the maps go from $E_{\alpha_p}$ to $E_{\alpha_0}$;
                        \item the maps are degree-$q$ \textit{chain maps}, i.e. they are `true' maps of complexes and respect the differentials.
                    \end{enumerate}

                    However, when using a twisting cochain to construct a total differential (as at the end of \cref{ssub:vector-bundles-and-graded-vector-spaces}), we get the same result whether we use this definition or a previous one.

                    Generally, the definition has evolved over time to become a more general construction, though it is interesting to consider what motivates these specific changes.
                \end{note}
                
                Next we set
                \[
                    \hat{\C}^p(\underline{U},\End^q(E)) = \smashoperator{\prod_{\substack{(\alpha_0\ldots\alpha_p)\text{ s.t.}\\U_{\alpha_0\ldots\alpha_p}\neq\varnothing}}}\End^q(E)|_{U_{\alpha_0\ldots\alpha_p}}
                \]
                and define our deleted \v{C}ech differential \textit{almost} exactly as before:
                \begin{align*}
                    \hat{\delta}\colon\hat{\C}^p(\underline{U},\End^q(E)) &\to \hat{\C}^{p+1}(\underline{U},\End^q(E))\\
                    (\hat{\delta}c)_{\alpha_0\ldots\alpha_{p+1}} &= \sum_{i=1}^p(-1)^ic_{\alpha_0\ldots\hat{\alpha_i}\ldots\alpha_{p+1}}|_{U_{\alpha_0\ldots\alpha_{p+1}}},
                \end{align*}
                so the sum only goes from $i=1$ to $p$, missing out both $i=0$ \textit{and} $i=p+1$.
                This is a natural modification to make to our differential, since we need to end up with a map from $E_{\alpha_p}$ to $E_{\alpha_0}$.

                We can further make this bicomplex into an $\mathcal{O}_X$-module by defining the obvious multiplication:
                \[
                    (c^{p,q}\cdot d^{r,s})_{\alpha_0\ldots\alpha_{p+r}} = (-1)^{qr}c^{p,q}_{\alpha_0\ldots\alpha_{p}}d^{r,s}_{\alpha_p\ldots\alpha_{p+r}}.
                \]

                \begin{definition}[Holomorphic twisting cochain]
                    A \textit{holomorphic twisting cochain} is an element $\mathrm{a}=\sum_{k\geqslant0}\mathrm{a}^{k,1-k}$ of total degree $1$ in $\hat{\C}^p(\underline{U},\End^q(E))$ such that
                    \begin{enumerate}
                        \item $\mathrm{a}^{0,1}_\alpha=\d_\alpha$;
                        \item $\hat{\delta}\mathrm{a}+\mathrm{a}\cdot\mathrm{a}=0$. \qedhere
                    \end{enumerate}
                \end{definition}

                \begin{note}
                    With this definition there is no requirement for $\mathrm{a}^{1,0}$ to be the identity, but this is simply because we put it in the definition of a \emph{twisted resolution} instead.
                \end{note}

                \begin{definition}[Holomorphic twisted resolution]
                    Let $\mathcal{F}$ be a sheaf of $\mathcal{O}_X$-modules on $X$.
                    Then a \textit{holomorphic twisted resolution of $\mathcal{F}$} is a triple $(\underline{U},E^\bullet,\mathrm{a})$ such that the following conditions are satisfied.
                    \begin{enumerate}
                        \item $\underline{U}=\{U_\alpha\}$ is a locally-finite open Stein cover of $X$.
                        \item $E^\bullet=(E^\bullet_\alpha,\d_\alpha)$ is a collection\footnote{There is a lot to gain from noting that $E^\bullet$ is more than just a collection: using the nerve of the cover we can realise it as a simplicial object of some sort.} of local locally-free resolutions of $\mathcal{F}$ over each $U_\alpha$ of globally-bounded length.\footnote{That is, each $E^\bullet_\alpha$ is a resolution of $\mathcal{F}|_{U_\alpha}$ by locally-free $\mathcal{O}_{U_\alpha}$-modules. Further, there exists some $B\in\mathbb{N}$ such that every $E^\bullet_\alpha$ is of length no more than $B$.}
                        \item $\mathrm{a}$ is a holomorphic twisting cochain \textit{over $\mathcal{F}$}.
                            This means that we have the following commutative diagram:
                            \begin{equation*}
                                \begin{tikzcd}[column sep=small]
                                    E^\bullet
                                        \ar{rr}{\mathrm{a}^{1,0}}
                                        \ar{dr}
                                    &
                                    &E^\bullet
                                        \ar{dl}\\
                                    &\mathcal{F}&
                                \end{tikzcd}
                            \end{equation*}
                        \item On degenerate simplices of the form $\alpha=(\alpha_0\ldots\alpha_p)$ with $\alpha_i=\alpha_{i+1}$ for some $i$, we have that $\mathrm{a}^{1,0}_\alpha=\id$ and $\alpha^{k,1-k}=0$ for $k>1$.\qedhere
                    \end{enumerate}
                \end{definition}
                This last condition is the identity condition that we've seen before (ensuring that we take values in some $\GL(n)$), since requiring $\mathrm{a}^{k,1-k}$ to be zero for $k>1$ just means that we want $\mathrm{a}^{1,0}$ to be the identity `on the nose'.

                \begin{note}
                    We have actually already talked about existence for a holomorphic twisting resolution in the case of $\mathcal{F}$ being coherent: this is simply \cite[Lemma~8.13]{Toledo:1976gy} again, or \cite[Lemma~2.4]{Toledo:1978tq}, which actually shows an even tighter result using the Hilbert syzygy theorem (namely that we can ensure that our global length bound $B$ is no more than the dimension of $X$).
                \end{note}

            % subsection twisting-resolutions (end)

        \section{Twisted complexes over DG categories}\label{sec:dg-categories}

            \subsection{Definitions and motivation}

                A good summary and explanation of some of the ideas found in \cite{Bondal:1991un}, and how they fit into the $A_\infty$- and $\infty$-category setting, is \cite[§4]{Faonte:2015vc}.

                \bigskip

                One of the motivations behind the definition of \emph{twisted complexes over a dg-category} found in \cite{Bondal:1991un} is the following: given a dg-category $\mathcal{A}$, what is the `smallest' dg-category $\mathcal{A'}$ into which $\mathcal{A}$ embeds such that we can define a shift and functorial cones in $\mathcal{A'}$.
                It turns out that $\mathcal{A'}$ is exactly the category of twisted complexes over $\mathcal{A}$.
                Further, if $\mathcal{A}$ is \emph{pretriangulated}\footnote{See \cite[§1]{Bondal:1991un}.} then this embedding is a quasi-equivalence\footnote{That is, it induces an equivalence of the homotopy categories} of dg-categories.
                This lets us `pull back' the shift and cones from $\mathcal{A'}$ to ones in $\mathcal{A}$, and it turns out that the homotopy category of $\mathcal{A}$ is triangulated with this shift functor and these cones.

                We refer the interested reader to the original paper for thorough definitions of this category and formalisations of the above suggestive motivations, and give only the definition of a twisted complex to see how it relates to what we have previously discussed.

                \begin{definition}[Twisted complex]
                    Let $\mathcal{A}$ be a dg-category.
                    Then a \emph{twisted complex $\mathfrak{C}$ over $\mathcal{A}$} is a collection
                    \[
                        \mathfrak{C} = \{(E_i\in\mathcal{A})_{i\in\mathbb{Z}},\,\, q_{ij}\colon E_i\to E_j\}
                    \]
                    where
                    \begin{itemize}
                        \item only finitely many of the $E_i$ are non-zero;
                        \item the $q_{ij}$ are morphisms of degree $i-j+1$, satisfying $\d q_{ij}+\sum_s q_{sj}q_{is}=0$.\qedhere
                    \end{itemize}
                \end{definition}

            \subsection{The relation to holomorphic twisting cochains}

                The twisting cochains of Toledo and Tong, as defined in \cref{sec:holomorphic-twisting-cochains}, are a specific example of the twisted complexes of Bondal and Kapranov, as we now show.

                Using the notation from \cref{ssub:twisting-resolutions}, we start with the data of a graded holomorphic vector bundle $\mathcal{E}_\alpha$ on every $U_\alpha$ and write $\mathcal{E}=\{\mathcal{E}_\alpha\}$.
                Let $B=\hat{\C}^\bullet(\underline{U},\mathcal{O}_X)$ and $E_0=\hat{\C}^\bullet(\underline{U},\mathcal{E})$.
                Now, over each $U_\alpha$ we can consider the graded $\mathcal{O}_{U_\alpha}$-module $\mathcal{E}_\alpha$ as a complex $\mathcal{E}_\alpha^\circ$ with trivial differential, and so $E_0$ is a $B$-dg-module.

                If we then define $E_i=0$ for $i\neq0$ we can ask what it means for $\{E_i\}_{i\in\mathbb{Z}}$ to have the structure of a twisted complex: we need to provide a degree-$1$ $B$-linear endomorphism $\mathrm{a}=q_{00}$ of $E_0$ such that $\d\mathrm{a}+\mathrm{a}\mathrm{a}=0$.
                But the dg-algebra $\End_B(E_0)$ is exactly\footnote{\cite[§1]{OBrian:1981vs} shows that we get a module structure on $\hat{\C}^\bullet(\underline{U},\mathcal{E}^\circ)$ over the algebra $\hat{\C}^\bullet(\underline{U},\End^\circ(\mathcal{E}))$} $\hat{\C}^\bullet(\underline{U},\End(\mathcal{E})^\circ)$.
                Decomposing $\mathrm{a}$ into $\mathrm{a}^{k,1-k}\in\hat{\C}^k(\underline{U},\End^{1-k}(\mathcal{E}))$ we recover exactly the condition of Toledo and Tong.
                Thus any holomorphic twisting cochain gives us a twisted complex in the dg-category of $B$-dg-modules.

                It is \emph{not} the case, however, that by picking the `right' dg-category $\mathcal{A}$ we recover the definition of holomorphic twisting cochains from that of twisted complexes.
                In particular, twisting cochains are twisted complexes with only \emph{one} non-zero object.
                Further, this object $E_0$ is not an arbitrary $B$-module: it comes from $\hat{\C}^\bullet(\underline{U},\mathcal{E})$, where $\mathcal{E}$ is graded.

            \subsection{DG-enhancement}\label{sub:dg-categories-from-wei}

                The idea of holomorphic twisting cochains can be used to construct a dg-enhancement of the derived category of perfect complexes of $\mathcal{O}_X$-modules on $X$, as shown in \cite{Wei:2016ip}.
                There is only one difference in the definition when compared to that of Toledo and Tong: the identity condition on $\mathrm{a}_{\alpha\alpha}^{1,0}$ holds \emph{only up to chain homotopy}.
                In a sense this is a more natural requirement, and it also guarantees that we can construct mapping cones in the category of twisted cochains: see \cite[Remark~2.15]{Wei:2016ip}.

                Recall that a complex $S^\bullet$ of $\mathcal{O}_X$-modules on a locally ringed space $(X,\mathcal{O}_X)$ is \emph{perfect} if, for any $x\in X$, there exists a bounded complex $\mathcal{E}_{U_x}^\bullet$ of vector bundles on some open neighbourhood $U_x$ of $x$, together with a quasi-isomorphism \mbox{$\mathcal{E}_{U_x}^\bullet\congto S^\bullet|_{U_x}$}.

                Let $\mathrm{Qcoh}_\mathrm{perf}(X)$ be category of perfect complexes of quasi-coherent sheaves; $\mathrm{D}_\mathrm{perf}(\mathrm{QCoh}(X))$ be the triangulated subcategory (of the derived category of complexes of quasi-coherent sheaves) consisting of perfect complexes; and $\mathrm{D}_\mathrm{perf}(X)$ be the triangulated subcategory (of the derived category of complexes of $\mathcal{O}_X$-modules) consisting of perfect complexes.

                \begin{note}
                    When working in the complex-analytic setting we need to use \emph{Fréchet quasi-coherent sheaves}: see \cite[Remark~6.4]{Wei:2016ip}.
                \end{note}

                One of the main results of \cite{Wei:2016ip} is the construction of a sheafification functor
                \begin{equation*}
                    \mathcal{S}\colon\mathrm{Tw}_\mathrm{perf}(X)\to\mathrm{QCoh}_\mathrm{perf}(X)
                \end{equation*}
                that, under `reasonable conditions', induces an equivalence of categories
                \begin{equation*}
                    \mathcal{S}\colon\operatorname{Ho}\mathrm{Tw}_\mathrm{perf}(X)\to\mathrm{D}_\mathrm{perf}(\mathrm{QCoh}(X))
                \end{equation*}
                and (under further conditions) an equivalence
                \begin{equation*}
                    \mathcal{S}\colon\operatorname{Ho}\mathrm{Tw}_\mathrm{perf}(X)\to\mathrm{D}_\mathrm{perf}(X).
                \end{equation*}

                This gives us our desired dg-enhancement of the derived category of perfect complexes of $\mathcal{O}_X$-modules on $X$.

        \section{A-infinity categories}

            \subsection{References}

                Here we cheat: we refer the reader to two different sources that cover various aspects of the idea of twisted complexes in the setting of $A_\infty$-categories.
                \begin{enumerate}[(i)]
                    \item \cite{Keller:2001tu}: This talks about twisted complexes in the sense of Bondal and Kapranov and shows how they are equivalently characterised by being the objects through which we factor the Yoneda embedding from an $A_\infty$-category $\mathcal{A}$ to the category of $\mathcal{A}$-modules.
                    \item \cite{Faonte:2015vc}: There is a construction called the \emph{simplicial nerve} for an $A_\infty$-category that mirrors the classical nerve construction.
                        Using the idea of pretriangulated dg-categories (that comes from the definition of twisted complexes) found in \cite{Bondal:1991un}, this paper shows that the simplicial nerve of $\mathcal{A}$ is a stable $\infty$-category (in the sense of \cite{Lurie:2016tv}) whenever $\mathcal{A}$ is pretriangulated.
                        It is interesting to compare this construction to that found in \cite[1.3.1.6]{Lurie:2016tv}.
                \end{enumerate}

    %%% Bibliography %%%

    \printbibliography

\end{document}
