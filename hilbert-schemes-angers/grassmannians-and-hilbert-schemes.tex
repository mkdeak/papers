\documentclass[10pt]{article}


% Preamble

\usepackage{charter}
\usepackage{amsmath,amssymb,amsthm}
\usepackage{titlesec}
\usepackage{embedfile}
\usepackage[margin=1.2in]{geometry}
\usepackage{url,hyperref}
\usepackage[utf8]{inputenc}

%% Table of contents

\setcounter{tocdepth}{2}

%% Section titles

\titleformat{\section}{\Large\bfseries\filcenter}{\thesection}{0.7em}{}[]
\titleformat{\subsection}{\large\bfseries}{\thesubsection}{0.5em}{}[]
\titleformat{\subsubsection}[runin]{\bfseries}{\thesubsubsection}{0.5em}{}[]
\titlespacing*{\subsubsection}{0pt}{.5\baselineskip}{1em}


%% Numbering

\numberwithin{equation}{subsubsection}

%% Title

\title{Grassmannians and Hilbert schemes}
\author{Timothy Hosgood}

%% Shortcuts

\DeclareMathOperator{\G}{G}
\DeclareMathOperator{\V}{\mathbb{V}}
\DeclareMathOperator{\sgn}{sgn}
\DeclareMathOperator{\Pl}{Pl}
\DeclareMathOperator{\codim}{codim}

\newcommand{\N}{\mathbb{N}}
\renewcommand{\P}{\mathbb{P}}
\newcommand{\A}{\mathbb{A}}
\renewcommand{\k}{\mathrm{k}}
\renewcommand{\Im}{\mathrm{Im}\,}
\newcommand{\Hilb}[1]{\mathrm{Hilb}^{#1}\,}


% Document

\begin{document}
    
    \embedfile{\jobname.tex}
    
    \maketitle
    \abstract{Notes taken at a masterclass given by Laurent Evain at Angers on the 13-15th December 2016. More likely than not that some mistakes have been introduced during typesetting and important points missed: please email \href{mailto:timhosgood@gmail.com}{\url{timhosgood@gmail.com}} with any corrections.}
    \tableofcontents

    \subsubsection{Warning.} We are sometimes sloppy with defining notation, but hopefully in an understandable way. Usually if any notation isn't defined it has been carried over from a previous section.
                
    \section{Grassmannians}
    
            \subsubsection{Notation.} $\k$ is an algebraically-closed field; the natural numbers include zero; $\{*\}$ is an arbitrary singleton set; $\P^n$ is projective $n$-space over $\k$; $\A^n$ is affine $n$-space over $\k$; $\langle v_i \mid i\in I\rangle$ means the subspace generated by the vectors $\{v_i\}_{i\in I}$; $\{e_1,\ldots,e_n\}$ is the canonical basis of $k^n$; $V^*$ is the dual space of a vector space $V$, and the dual basis of $\{v_1,\ldots,v_n\}$ is written $\{v_1^*,\ldots,v_n^*\}$; $S_n$ is the symmetric group on $n$ elements; we write $S\subset T$ to mean $S\subseteq T$ (and write $S\subsetneq T$ if $S\subset T$ but $S\neq T$).

        \subsection{Set-theoretic introduction}
        
            \subsubsection{The idea.} Projective space can be thought of as the set of lines through the origin on a vector space under some equivalence relation. We can generalise this by looking at linear subspaces of some fixed arbitrary dimension.
        
            \subsubsection{Definition.} For a $\k$-vector space $W$ and natural number $p\in\N$ we define the \textit{$p$-Grassmannian associated to $W$} as the set of dimension-$p$ subspaces of $W$, i.e. \[\G(p,W)=\{V\subset W \mid \dim V=p\}.\]
                For $q\in\N$ we define $\G(p,q)=\G(p,\k^q)$.
            
            \subsubsection{Trivial examples.} $\G(0,W)=\G(p,p)=\{*\}$.
            
            \subsubsection{Less trivial examples.} $\G(1,q)=\G(q-1,q)\cong\P^{q-1}$. To see this, note that an element of e.g. $\G(q-1,q)$ is given uniquely, up to multiplication by a non-zero scalar $\lambda\in\k^\times$, by a linear form $L=a_0x_0+\ldots+a_{q-1}x_{q-1}$. This gives us the bijection $L\leftrightarrow[a_0:\ldots:a_{q-1}]\in\P^{q-1}$.
            
            \subsubsection{An explicit example.} An element of $\G(2,4)$ is determined by a pair of linear forms, say $L=\sum_{i=0}^3 a_ix_i$ and $M=\sum_{i=0}^3 b_ix_i$. We can calculate the determinants $D_{ij}$ of the six $2\times2$-minors of the $2\times4$-matrix whose rows are $(a_j)$ and $(b_j)$ (e.g. $D_{01}=a_0b_1-a_1b_0$). These are determined up to a constant, and are invariant under choosing a different pair of linear forms to represent the same element. It can be checked that $D=D_{01}D_{23}-D_{02}D_{13}+D_{03}D_{12}$ satisfies $D=0$. In fact, the map $\varphi\colon\G(2,4)\to\P^5$ given by $V\mapsto(D_{ij})$ has image given exactly by the vanishing of $D$, i.e. \[\Im\varphi = \{[D_{01}:D_{02}:D_{03}:D_{12}:D_{13}:D_{23}]\in\P^5 \mid D=0\}.\]
            
            \subsubsection{A trick.}\label{a-trick} Let $V=\langle e_1,e_2+e_3\rangle\subset W=k^4$. Then $V$ is defined by the vanishing of $e_4$ and $e_2-e_3$, i.e. by $e_4^*$ and $e_2^*-e_3^*$ in $W^*$. Taking the wedge product gives us $e_4^*\wedge (e_2^*-e_3^*) = -e_2^*\wedge e_4^* + e_3\wedge e_4^*$ from which we can read off $D_{24}=-1$ and $D_{34}=1$. This is explained in the next section.
        
        \subsection{Dual view}
        
            \subsubsection{The idea.} Look at the generators of a subspace $V$ rather than its equations.
            
            \subsubsection{Explicit example.} With $V$ as in \ref{a-trick} we associate $V\leftrightarrow e_1\wedge (e_2+e_3)$.
            
            \subsubsection{Speedy notation.} Write $e_I = e_{i_1}\wedge\ldots\wedge e_{i_r}$ where $I\subset\{1,\ldots,n\}$ and $i_1<\ldots<i_r$. Given such an $I$, write $J=\{1,\ldots,n\}\setminus I$ and $\sigma_{IJ}=(I\,J)\in S_n$. Given $\lambda e_I$ for some $\lambda\in\k^\times$ we write $p_I=\lambda$.
            
            \subsubsection{Another explicit example.} Look at $V\in\G(2,3)$ defined by $\sum_{i=1}^3 q_ie_i^*$ where $q_i=i$, so that $V$ has a basis $\{2e_1-e_2,3e_1-e_3\}$. Then \[(2e_1-e_2)\wedge (3e_1-e_3) = -2e_{1,3}+3e_{1,2}+e_{2,3}.\] Notice that, up to a sign, $p_I=q_J$ where $J=\{1,2,3\}\setminus I$.
            
            \subsubsection{Proposition/definition.} Let $V\subset W\cong\k^q$ be of dimension $p$ with basis $\{b_1,\ldots,b_p\}$ and defined by equations $\{\varphi_1,\ldots,\varphi_{q-p}\}$. Write $\bigwedge_{i=1}^pb_i = \sum p_Ie_I=p_V$ where the sum is taken over all $I=(i_1,\ldots,i_p)$ with $i_1<\ldots<i_p$. Similarly, write $\bigwedge_{j=1}^{q-p}\varphi_j=\sum q_Je_J^*=q_V$, where $J=\{1,\ldots,q\}\setminus I$ is also ordered with $j_1<\ldots<j_{q-p}$. Then the $p_I$ and $q_J$ are called \textit{Plücker coordinates} and are related by $p_I=\sgn\sigma_{IJ}q_I$.
            
            \subsubsection{Definition.} The \textit{Plücker embedding} is the morphism $\Pl\colon\G(p,q)\to\P^{N}$, where $N=\binom{q}{p}-1$, given by $V\mapsto(p_I(V))$, i.e. by taking the Plücker coordinates.
        
        \subsection{Nested spaces}
        
            \subsubsection{The idea.} We can see if a point lies on a hyperplane by taking the equations that define the hyperplane and evaluating them at the point, i.e. we take the dual of one subspace and evaluate it on the other.
            
            \subsubsection{Nesting.} Two subspaces $U,V\subset W$ are said to be \textit{nested} if either $U\subset V$ or $V\subset U$.
        
            \subsubsection{Duality.} There is a canonical isomorphism $\P\left(\bigwedge^p W\right)\cong\P\left(\bigwedge^{q-p} W^*\right)$. Using the $p_I(V)$ as Plücker coordinates means we are working in the left-hand side; using the $q_J(V)$ means we are working in the right-hand side. We are free to switch between the two to describe $V$.
            
            \subsubsection{Explicit example.} Let $U = \langle e_1,e_2+ae_3\rangle,V = \langle e_1,e_2+be_4\rangle\subset\k^4$ for some $a,b\in\k$. Then $U=V$ if and only if $a,b=0$. Now we calculate the Plücker coordinates: $p_V = e_{1,2}+ae_{1,3}$ and $q_V = -be_{2,3}^*-e_{3,4}^*$. We `evaluate' $q_V$ at $p_V$ by taking their product: \[p_V\cdot q_V = be_1\wedge e_3^*-abe_1\wedge e_2^*+ae_1\wedge e_4^*.\] This product is calculated by using the rules that $e_i\cdot e_i^* = e_i^*(e_i) = 0$ and the linearity and anticommutativity of the wedge product (and is further explained below).
            
            \subsubsection{The product.} Generally we define the above product on the level of tensors (since the exterior/wedge product is a quotient of the tensor product), i.e. \[\varphi\colon(\wedge^mW\otimes\wedge^nW^*)\to(\wedge^{m-1}W\otimes\wedge^{n-1}W^*)\] and $p_V\cdot q_V=\varphi(p_V\otimes q_V)$.
            
            \subsubsection{Proposition.} Let $U,V\subset W$. Then $\varphi(p_V\otimes q_U)=0$ if and only if $U$ and $V$ are nested.
            
            \subsubsection{Corollary.} There exists some projective variety $Q\subset\P^N$ defined by quadrics (called the \textit{Plücker relations}) such that $\Im(\Pl)\subset Q$.
            \begin{proof}
                $V\in\G(p,W) \iff V\subset V \iff \varphi(p_V\otimes q_V)=0$. This last equation is of bi-degree $(1,1)$ but, up to a sign, $p_V=q_V$, so it can be thought of as being degree-$2$ in $p_V$.
            \end{proof}
            
            \subsubsection{Theorem.} $\Im(\Pl)=Q$.
        
        \subsection{Algebraic structure}
        
            \subsubsection{Idea.} We give the Grassmannian a natural algebraic structure by exhibiting it as an object with a universal property.
            
            \subsubsection{More concrete idea.} Find an equivalence between morphisms $X\to\G(p,q)$ and bundles on $X$, noting that $\G(p,q)$ has a universal bundle.
            
            \subsubsection{Definition.} A \textit{vector bundle (of rank $r$)} on a vector space $X$ is an object/morphism pair $(\varphi\colon F\to X)$ such that \[\forall p\in X \quad \exists U_p\underset{\text{open}}{\subset}X \quad \text{s.t.} \quad p\in U_p\text{ and }\varphi^{-1}(U_p)\cong U_p\times\k^r\]
                and we write $F_p=\varphi^{-1}(p)$ to mean the \textit{fibre over $p$}.
                \textbf{n.b.} the actual definition is slightly more involved, since we need to ensure that the fibres glue together nicely, but we refer the reader elsewhere for the full definition.
            
            \subsubsection{Example.}\label{bundle-example} Consider $\k^3\setminus\{0\}\times\k^3$ with coordinates $(a_1,a_2,a_3,x_1,x_2,x_3)$. Define a bundle $F$ on $\k^3\setminus\{0\}$ by setting $F_{(a_1,a_2,a_3)}=\{a_1x_1+a_2x_2+a_3x_3=0\}$. This is a rank-$2$ bundle.
            
            \subsubsection{Definition.} Given a continuous map $f\colon X\to Y$ and a bundle $F\to Y$ we define a bundle $f^*F$ on $X$, called the \textit{pullback $F$ under $f$}, by $(f^*F)_x=F_{f(x)}$
            
            \subsubsection{Example.} Let $F$ be the bundle from \ref{bundle-example} and define $f\colon\k\to\k^3\setminus\{0\}$ by $t\mapsto(t^2,t^3,1)$. Then $(f^*F)_t=\{t^2x_1+t^3x_2+x_3=0\}$.
            
            \subsubsection{Definition.} Define a rank-$p$ bundle $U\to\G(p,W)$ by $U_V=V$. This is called the \textit{universal bundle of the Grassmannian}.
            
            \subsubsection{Theorem.} Specifying a morphism $X\to\G(p,W)$ is equivalent to specifying a rank-$p$ bundle $F\to X$ such that $F\subset X\times W$.
            
            \subsubsection{Theorem.} Given a rank-$p$ bundle $F\to X$ there exists a unique $f\colon X\to\G(p,W)$ such that $f^*U\cong F$ (as bundles), and this $f$ is exactly the morphism given by $x\mapsto F_x$.
    
    \section{Hilbert schemes}
    
            \subsubsection{Notation.} Given a graded ring $R$ we write $R_d$ to mean the degree-$d$ part; given some ideal $I\triangleleft R$ we write $I_d$ to mean the degree-$d$ part; we write $R^{(i,j)}=\k[x_i,\ldots,x_j]$.
            
            \subsubsection{Warning.} We ignore many scheme-theoretic details (i.e. assume reducedness) and don't always rigorously define all of our ideas (for the sake of time), especially commonly-known ones. From hereon in these notes are intended more for somebody with a reasonable (but by no means large) working knowledge of algebraic geometry.
    
        \subsection{Ideals}
        
            \subsubsection{Question.}\label{identification-question} How do we distinguish similar objects in projective space?
            
            \subsubsection{Simple cases.} We can consider vector spaces up to dimension, and discrete sets of points up to cardinality, i.e. $\{p_1,\ldots,p_r\}\sim\{q_1,\ldots,q_s\}\iff r=s$.
            
            \subsubsection{Idea.} We know that we can associate to each subset $X\subset\P^n$ some ideal $I(X)\triangleleft R^{(0,n)}$, and vice versa. But the ideal might contain more (scheme-theoretic) information than the subset. So we want to find some property that is shared by `similar' ideals, like we have dimension for vector spaces and cardinality for discrete sets. This will be the \textit{Hilbert polynomial}.
            
            \subsubsection{Example.} Consider $X=\{0\}\subset\A^1$. Then $I(X)=(x)\triangleleft\k[x]$. Then `set-theoretically' $(x)=(x^2)=X$, but `scheme-theoretically' $I(X)=(x)\neq(x^2)$.
            
            \subsubsection{Explicit example/definition.} Let $p=(a,b)\in\A^2$. Then we have the following ideals: the affine ideal $I_*(p)=(x-a,y-b)\triangleleft R^{(0,1)}$ and the projective ideal $I(p)=(x-az,y-bz)\triangleleft R^{(0,2)}$. Here \[I_2(p)=\big(x(x-az),y(x-az),z(x-az),x(y-bz),y(y-bz),z(y-bz)\big)\] and this is a subspace of $R^{(0,2)}_2=\k[x^2,y^2,z^2,xy,xz,yz]$. It can be checked that $\dim R^{(0,2)}_2/I_2(p)=1$ and thus $\dim I_2(p)=5$, i.e. $I_2(p)\in\G(5,R^{(0,2)}_2)$.
            
            \subsubsection{Proposition.} If $X=\{p_1,\ldots,p_4\}\subset\P^2$ consists of collinear points then \mbox{$\codim(I_2(X),S_2)=3$.}
            
            \begin{proof}
                Suppose the line is $y=0$. Then the conic meets all four points and so contains the line $y=0$. Thus the possible equations of such a conic are of the form $y(ax+by+cz)$, thus $\dim I_2=3$.
            \end{proof}
            
            \subsubsection{Proposition.} If $X=\{p_1,\ldots,p_4\}\subset\P^2$ consists of points in general position then \mbox{$\codim(I_2(X),S_2)=4$.}
            
            \subsubsection{Problem.} We want to be able to consider ideals as elements of a Grassmannian, but depending on the geometry of our subsets the ideals lie in different Grassmannians. Can we find a uniform degree which works for all $X$?
        
        \subsection{Limits}
            
            \subsubsection{Question.} Let $X(t)=\{(0,0),(t,0)\}\subset\A^2$. What is a `good' definition of $\lim_{t\to0}X(t)$?
            
            \subsubsection{Bad answer.} Naïvely, $\lim_{t\to0}X(t)=\{(0,0)\}$. This is not a `continuous' notion, since we collapse two points to one. The ideal associated to this set is $(x,y)$.
            
            \subsubsection{Better answer.} Work with ideals, so that we retain the differential information: $I(X(t))=(y,(x-t)x)\xrightarrow{t\to0}(y,x^2)$. This has codimension $2$ since $\k[x,y]/(y,x^2)$ has $\{1,x\}$ as a $\k$-basis.
            
            \subsubsection{Technical lemma.} Let $f\in\k[d]$ be a polynomial, $C\subset\P^2$ a curve, and $x\in C$ a point such that $C$ is smooth near $x$. Suppose we have some closed $Z\subset C\setminus\{x\}\times\P^n$ such that $h_{I(Z_q)}=f$ for all $q\in C$, where $Z_q$ is the fibre over $q$ of the bundle $C\times\P^n$. Then there exists a unique $Z'\subset C\times\P^n$ such that $h_{I(Z'_x)}=f$ and $Z'$ extends $Z$. In fact, $Z'$ is exactly the (scheme) closure of $Z$.
            
            \subsubsection{Definition.} An ideal $I\triangleleft R^{(0,n)}$ is \textit{saturated} if $\forall f\in R^{(0,n)}\,\,\big[x_0f,\ldots,x_nf\in I\implies f\in I\big]$.
            
            \subsubsection{Informal definition.} We make the following temporary definitions:
                \begin{gather*}
                    \Hilb{m}(\A^2)=\{I\triangleleft k[x,y] \mid \codim(I,k[x,y])=m\}\\
                    \Hilb{m}(\P^2)=\{I\triangleleft k[x,y,z] \mid I\text{ homogeneous saturated, }\codim(I_d,k[x,y,z]_d)=m\text{ for all sufficiently large }d\}                    
                \end{gather*}
        
        \subsection{Hilbert polynomials}
        
            \subsubsection{Proposition/definition.} If $I\triangleleft R^{(0,n)}$ is homogeneous and saturated then there exists a polynomial $h_I(d)$ such that $\codim(I_d,R^{(0,n)}_d)=h_I(d)$ for all sufficiently large $d$. This $h_I$ is called the \textit{Hilbert polynomial}.
            
            \subsubsection{Explicit example.} $\P^1\subset\P^n$ has Hilbert polynomial $h_{I(\P^1)}(d)=d+1$ since $I(\P^1)=(x_2,\ldots,x_n)$ and so $k[x_0,\ldots,x_n]_d/I_d(\P^1)$ has $\{x_0^d,x_0^{d-1}x_1,\ldots,x_1^d\}$ as a basis. Note that that $h_{I(\P^1)}$ doesn't depend on $n$.
            
            \subsubsection{Explicit example.} $\P^2\subset\P^n$ has Hilbert polynomial $h_{I(\P^2)}=\binom{d+2}{d}$ by a similar argument.
            
            \subsubsection{Answer.} Recalling \ref{identification-question} we say that $I(X)\sim I(Y)\iff H_{I(X)}=H_{I(Y)}$.
            
            \subsubsection{Question.} How do we construct $\{I\triangleleft R^{(0,n)}\mid I\text{ hom. sat., }h_I=f\}$?
        
        \subsection{Uniformisation}
        
            \subsubsection{Question.} Which polynomials are Hilbert polynomials of some ideal?
            
            \subsubsection{Partial answer.} $f_n(d)=\binom{d+n}{n}$ is the Hilbert polynomial of $\P^n$.
            
            \subsubsection{Proposition.} If $r_1\geqslant\ldots\geqslant r_k$ then \[f_{r_1}(d)+f_{r_2}(d-1)+\ldots+f_{r_k}(d-k+1)\] is the Hilbert polynomial of the unions of the $\P^{r_i}$ (though there are lots of subtle points here).
            
            \subsubsection{Theorem.} Every Hilbert polynomial is of this form.
            
            \subsubsection{Explicit example.} $3d+1=(d+1)+d+(d-1)+1$ and so is the Hilbert polynomial of the union of three lines and a point (and we can take this union in $\P^2$ it turns out).
            
            \subsubsection{Explicit non-example.} $d^2-d-4$ has a negative $d$ coefficient, and so can never be realised as a Hilbert polynomial.
            
            \subsubsection{Definition.} The number of terms in this sort of decomposition is called the \textit{Gotzmann number}.
            
            \subsubsection{Theorem.} Let $I\triangleleft R^{(0,n)}$ be saturated and homogeneous with Hilbert polynomial $h_I$ and Gotzmann number $r$. Then $\codim(I_d,S_d)=h_I(d)$ for all $d\geqslant n$. Further, this bound is optimal.
            
            \subsubsection{Definition.} As a set, we define $\Hilb{f}(\P^n)=\{I\triangleleft R^{(0,n)}\mid I\text{ hom. sat., }h_I=f\}$
            
            \subsubsection{Theorem.} Let $S=R^{(0,n)}$. Then the map
                \begin{align*}
                    \varphi \colon \Hilb{f}(\P^n) &\to \G(\dim S_r-f(r),S_r)\\
                    I &\mapsto I_r
                \end{align*}
                is injective. Further, $\Im\varphi$ can be defined by algebraic equations.
                
    \section{Applications}
    
        \subsection{Interpolation in degree one}
        
            \subsubsection{Theorem.}\label{first-interpolation-theorem} Given distinct $a_0,\ldots,a_d,b_0,\ldots,b_d\in\P^1$ there exists a unique $P\in\k[x]_d$ such that \mbox{$P(a_i)=b_i$} for all $i$.
            
            \subsubsection{Proof (idea).} Consider the map $\k[x]_d\to\k^{d+1}$ given by $P\mapsto(P(a_0),\ldots,P(a_d))$. Using the basis $\{1,x,x^2,\ldots,x^d\}$ we can write this map as a Vandermonde matrix, whose determinant is $\prod_{i\leqslant j}(a_i-a_j)\neq0$.
            
            \subsubsection{Proposition/definition.} Let $S=\k[x,y]$ and $p_1,\ldots,p_r\in\P^1$. Define \[S_d(-p_1\ldots-p_r)=\{f\in S_d \mid f(p_i)=0\,\,\forall i\}.\]
                Then $\codim(S_d(-p_1\ldots-p_r),S_d)=\min\{r,d+1\}$
    
        \subsection{Interpolation in degree two}
        
            \subsubsection{Question.} How many lines in $\P^2$ are there that go through three specified points?
            
            \subsubsection{Answer.} Generally there are none: $L=ax+by+cz$ has three conditions imposed on it and so reduces to $0$. Here, by `generally' we mean that the points are in general position.
            
            \subsubsection{Proposition.}\label{second-interpolation-theorem} Let $p_1,\ldots,p_r\in\P^2$ be in general position. Then \[\codim(S_d(-p_1\ldots-p_r),S_d)=\min(r,\dim S_d).\]
            
            \subsubsection{Example.} Let $d=2$ and $r=6$. Now $\dim S_2=6$, so we expect no conic to contain six given points. But what is $\dim S_2(-p_1\ldots-p_6)$? As in \ref{first-interpolation-theorem} we can consider where the determinant of a matrix $M$ vanishes. There exists a Zariski open subset $U\subset(\P^2)^6$ such that $M$ has maximal rank, and so it suffices to find one position of points where $M$ is invertible.
                Say the $p_i$ are such that three are collinear on the line $L$. Then by Bézout we know that any conic $C$ going through these three collinear points must contain $L$, and so must be the union of $L$ and another line.
                But if the remaining three points aren't collinear then the conic cannot pass through all six points.
                
            \subsubsection{Proof (of \ref{second-interpolation-theorem}) (idea).} !!!
            
        \subsection{Multiplicities}
        
            \subsubsection{Definition.}
            
            \subsubsection{Dimension one.}
            
            \subsubsection{Dimension two.}
            
            \subsubsection{Example.}
            
            \subsubsection{Example.}
        
        \subsection{Quadratic transformations}
        
        \subsection{Hilbert's 14th problem}
    
\end{document}