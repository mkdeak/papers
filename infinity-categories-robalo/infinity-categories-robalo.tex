\documentclass{article}

\usepackage[top=1.2in,bottom=1.2in,left=1.8in,right=1.8in]{geometry}
\usepackage{amsmath}
\usepackage{amssymb}
\usepackage{charter}
\usepackage{titlesec}
\usepackage{datetime}
\newdate{date}{12}{06}{2017}

% Section titles

% \titleformat{\section}{\Large\bfseries\filcenter}{\thesection}{0.7em}{}[]
\titleformat{\section}{\large\bfseries}{\thesection}{0.5em}{}[]
\titleformat{\subsection}[runin]{\bfseries}{\thesubsection}{0.5em}{}[]

% Numbering

\numberwithin{equation}{subsection}

% Shortcuts

\renewcommand{\ss}[1]{\subsection{#1}}

% Title page

\title{Introduction to infinity categories}
\author{Talk by Marco Robalo at DAGIT 2017\\Typed by Timothy Hosgood}
\date{\displaydate{date}}

\begin{document}

    \maketitle

    \begin{abstract}
        These are a copy of my notes on a talk given by Marco Robalo at the seminar Derived Algebraic Geometry in Toulouse (DAGIT) 2017: the content is purely his; the mistakes are all mine.
    \end{abstract}

    \section{Motivation}

        \ss{Idea.}
            An \textbf{$\infty$-category} consists of
            \begin{itemize}
                \item objects;
                \item $1$-morphisms between objects;
                \item $n$-morphisms between $(n-1)$-objects (for $n\geqslant2$);
                \item composition laws for $n$-morphisms ($n\geqslant1$) defined up to higher morphisms;
                \item associativity of compositions up to homotopy.
            \end{itemize}

        \ss{Proto-example.} (Fundamental $\infty$-groupoid)
            For a CW-complex $X$ we have
            \begin{itemize}
                \item objects = points;
                \item $1$-morphisms = homotopies;
                \item $2$-morphisms = homotopies of homotopies;
                \item ... and so on.
            \end{itemize}

        \ss{Problem.}
            No direct definition that is operational and simultaneously close to our intuition/desire (infinitely many axioms!).

        \ss{Solution.}
            Find a model category whose objects serve as models for $\infty$-categories.

        \ss{Modelling.}
            Many classical examples:
            \begin{itemize}
                \item homotopy types can be modelled by topological spaces, simplicial sets, categories, etc.;
                \item homotopy theory of homotopy-commutative $\mathbb{Q}$-algebras can be modelled by dg-algebras;
                \item derived stacks can be modelled by simplicial presheaves.
            \end{itemize}

        \ss{Question.}
            Why so many models?

        \ss{Answer.}
            Dwyer-Kan localisation: every model category has an associated $\infty$-category that captures all the important information.

        \ss{Question.}
            If we have models then why care about $\infty$-categories?

        \ss{Answer.}
            Many reasons:
            \begin{itemize}
                \item not all $\infty$-categories have a model presentation;
                \item no `good enough' definition of functors that relate different models (need an $\infty$-functor between the associated $\infty$-categories);
                \item models for diagrams are not always given by diagrams of models;
                \item proofs and statements become `simpler'.
            \end{itemize}

    \section{Preliminary definitions}

    \section{Quasi-categories}

    \section{Simplicial nerve and rectification}

    \section{Homotopy colimits}

    \section{Localisation}

    \section{Presheaves and $\infty$-functors}

    \section{Presentability}

    \section{Symmetric monoidal $\infty$-categories}

    \section{Subtleties}

\end{document}
