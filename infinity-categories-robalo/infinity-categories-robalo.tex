\documentclass{article}

\usepackage[top=1.2in,bottom=1.2in,left=1.8in,right=1.8in]{geometry}
\usepackage{amsmath}
\usepackage{amssymb}
\usepackage{tikz-cd}
\usepackage{charter}
\usepackage{titlesec}
\usepackage{datetime}
\newdate{date}{12}{06}{2017}

% Section titles

% \titleformat{\section}{\Large\bfseries\filcenter}{\thesection}{0.7em}{}[]
\titleformat{\section}{\large\bfseries\filcenter}{\thesection}{0.5em}{}[]
\titleformat{\subsection}[runin]{\bfseries}{\thesubsection.}{0.5em}{}[]

% Numbering

\numberwithin{equation}{subsection}

% Shortcuts

\renewcommand{\ss}[1]{\subsection{#1}}
\newcommand{\Hom}{\mathrm{Hom}}
\newcommand{\sset}{\mathsf{sSet}}
\newcommand{\C}{\mathcal{C}}
\newcommand{\D}{\mathcal{D}}
\newcommand{\id}{\mathrm{id}}

% Title page

\title{Introduction to infinity categories}
\author{Talk by Marco Robalo at DAGIT 2017\\Typed by Timothy Hosgood}
\date{\displaydate{date}}

\begin{document}

    \maketitle

    \begin{abstract}
        These are a copy of my notes on a talk given by Marco Robalo at the seminar Derived Algebraic Geometry in Toulouse (DAGIT) 2017: the content is purely his; the mistakes are all mine.
    \end{abstract}

    \section{Motivation}

        \ss{Idea.}
            An \textbf{$\infty$-category} consists of
            \begin{itemize}
                \item objects;
                \item $1$-morphisms between objects;
                \item $n$-morphisms between $(n-1)$-objects (for $n\geqslant2$);
                \item composition laws for $n$-morphisms ($n\geqslant1$) defined up to higher morphisms;
                \item associativity of compositions up to homotopy.
            \end{itemize}

        \ss{Proto-example.} (Fundamental $\infty$-groupoid)
            For a CW-complex $X$ we have
            \begin{itemize}
                \item objects = points;
                \item $1$-morphisms = homotopies;
                \item $2$-morphisms = homotopies of homotopies;
                \item ... and so on.
            \end{itemize}

        \ss{Problem.}
            No direct definition that is operational and simultaneously close to our intuition/desire (infinitely many axioms!).

        \ss{Solution.}
            Find a model category whose objects serve as models for $\infty$-categories.

        \ss{Modelling.}
            Many classical examples:
            \begin{itemize}
                \item homotopy types can be modelled by topological spaces, simplicial sets, categories, etc.;
                \item homotopy theory of homotopy-commutative $\mathbb{Q}$-algebras can be modelled by dg-algebras;
                \item derived stacks can be modelled by simplicial presheaves.
            \end{itemize}

        \ss{Question.}
            Why so many models?

        \ss{Answer.}
            Dwyer-Kan localisation: every model category has an associated $\infty$-category that captures all the important information.

        \ss{Question.}
            If we have models then why care about $\infty$-categories?

        \ss{Answer.}
            Many reasons:
            \begin{itemize}
                \item not all $\infty$-categories have a model presentation;
                \item no `good enough' definition of functors that relate different models (need an $\infty$-functor between the associated $\infty$-categories);
                \item models for diagrams are not always given by diagrams of models;
                \item proofs and statements become `simpler'.
            \end{itemize}

    \section{Preliminary definitions}

        \ss{Category of simplices.}
            Write $\Delta$ to be the \textbf{category of simplices}:
            \begin{itemize}
                \item $\mathrm{ob}\Delta = \{[n]\}_{n\in\mathbb{N}}$ where $[n]=\{0<1<\ldots<n\}$ is the ordered set of natural numbers up to $n$;
                \item $\Hom_\Delta([m],[n])$ is the set of order-preserving maps from $[m]$ to $[n]$.
            \end{itemize}

        \ss{Simplicial notation.}
            We use the following notation:
            \begin{itemize}
                \item $\sset = \mathsf{Set}^{\Delta^\mathrm{op}} = \mathsf{Fun}(\Delta^\mathrm{op},\mathsf{Set})$;
                \item $\Delta[n] = \Hom_\Delta(-,[n])\in\sset$;
                \item $S_n = \Hom_\sset(\Delta[n],S)$ for $S\in\sset$;
                \item $\Lambda^i_n = \Delta[n]\setminus\{\text{interior and the face opposite the }j\text{-th vertex}\}$ is the \textbf{$i$-th horn}.
            \end{itemize}
            \[
                \Delta[2] =
                \begin{tikzcd}[row sep=1em,column sep=1em]
                    &1\ar{dr}\ar[Rightarrow]{d}&\\
                    0\ar{ur}\ar{rr}&\,&2
                \end{tikzcd}
                \qquad\Lambda_2^1 =
                \begin{tikzcd}[row sep=1em,column sep=1em]
                    &1\ar{dr}&\\
                    0\ar{ur}&\,&2
                \end{tikzcd}
            \]

        \ss{Nerve.}
            The \textbf{nerve} $N(\C)$ of a category $\C$ is the simplicial set with
            \begin{itemize}
                \item $n$-simplices given by $(X_0\xrightarrow{f_1}X_1\xrightarrow{f_2}\ldots\xrightarrow{f_n}X_n)$ in $\C$;
                \item boundary maps given by composition (or forgetting the first/last object and morphism for the two edge cases);
                \item degeneracy maps given by inserting the identity.
            \end{itemize}

        \ss{Note.}
            There is a set-bijection
            \[
                \{\text{functors }\C\to\D\}\simeq\{\text{simplicial maps }N(\C)\to N(\D)\}.
            \]

        \ss{Lemma.}
            There is an equivalence of categories $X\simeq N(\C)$ if and only if all \emph{inner} horns lift \emph{uniquely}.
            \[
                \begin{tikzcd}
                    \Lambda_n^i \rar \dar[hook] & X\\
                    \Delta[n] \urar[dashed,swap]{\exists!} &
                \end{tikzcd}
                \quad\text{for all }i\in\{1,\ldots,n-1\}
            \]
        \ss{Composition.}
            For example, ``$\Lambda_2^1$ gives composition''.
            \[
                \begin{tikzcd}[row sep=1em,column sep=1em]
                    &X_1\ar{dr}{f_2}&\\
                    X_0\ar{ur}{f_1}\ar[dashed]{rr}{\exists!f_2\circ f_1}&&X_2
                \end{tikzcd}
            \]

        \ss{Associativity.}
            As another example, ``$\Lambda_3^1$ gives associativity''.
            $X_0\xrightarrow{f_1}X_1\xrightarrow{f_2}X_2\xrightarrow{f_3}X_3$ in $\C$ corresponds to
            \[
                \Lambda_2^1\xrightarrow{(f_2,f_1)}N(\C)
                \quad\text{and}\quad
                \Lambda_2^1\xrightarrow{(f_3,f_2)}N(\C)
            \]
            which generate compositions
            \[
                \Delta[2]\xrightarrow{(f_2,f_1)}N(\C)
                \quad\text{and}\quad
                \Delta[2]\xrightarrow{(f_3,f_2)}N(\C).
            \]
            So $(f_3\circ f_2)\circ f_1 = f_3\circ(f_2\circ f_1)$ if and only if we can `fill the back face of the tetrahedron with vertices $X_0$, $X_1$, $X_2$, and $X_3$', i.e. if and only if we can extend $\Lambda_3^1\to N(\C)$ to $\Delta[3]\to N(\C)$.

        \ss{Summary.}
            The lifting property for inner horns ($0<i<n$) gives composition and associativity laws; for outer horns ($i=0,n$) it gives inverses.
            \[
                \begin{tikzcd}[row sep=1em,column sep=1em]
                    &X_0\ar[dashed]{dr}{\exists!}&\\
                    X_0\ar{ur}{\id_{X_0}}\ar{rr}{f_1}&&X_1
                \end{tikzcd}
            \]

        \ss{Kan complex.}
            If $X\sset$ is such that $X\simeq\mathrm{Sing}(T)$, where $\mathrm{Sing}(T)$ consists of singular simplices in a topological space $T$ (i.e. continuous maps $|\Delta^n|\to T$) then we call it a \textbf{Kan complex}.
            Note that $X$ is a Kan complex if and only if \emph{all} horns lift, but \emph{not necessarily} uniquely.

    \section{Quasi-categories}

    \section{Simplicial nerve and rectification}

    \section{Homotopy colimits}

    \section{Localisation}

    \section{Presheaves and $\infty$-functors}

    \section{Presentability}

    \section{Symmetric monoidal $\infty$-categories}

    \section{Subtleties}

\end{document}
